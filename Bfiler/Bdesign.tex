\input{setup}
\begin{document}
\begin{titlingpage}
\begin{center}

~ \\[3cm]

%\includegraphics[width=0.6\textwidth]{figurer/ASE}~\\[1cm]
\textsc{\LARGE Bilag 5}\\[1.5cm]

%\textsc{\Large Sundhedsteknologi}\\
%\textsc{\Large 3. semesterprojekt}\\[0.5cm]

\noindent\makebox[\linewidth]{\rule{\textwidth}{0.4pt}}\\
[0.5cm]{\Huge Design}
\noindent\makebox[\linewidth]{\rule{\textwidth}{0.4pt}}
\end{center}
\vfill
\begin{center}
{\large 19. december 2017}
\end{center}
\end{titlingpage}

\newpage
\tableofcontents*
\newpage

\chapter{Indledning}
I dette bilag forklares design af hardware og software  for synkerefleksmonitoren. På baggrund af HW-arkitekturen redegøres der, hvordan HW-blokkene, som indgår i HW-arkitekturen er designet samt deres funktion. Desuden indeholder afsnittet en kort beskrivelse af designovervejelser i forhold til de enkelte hardwareenheder. Figur \ref{blokflow} viser de komponenter, som skal anvendes for at synkerefleksmonitoren kan blive realiseret. Nogle af disse komponenter skal designes af gruppens medlemmer mens  resten er kommercielle komponenter, der anvendes pga. deres specifikationer. Disse kommercielle komponenter omfatter en Analog Discovery, en PC og en EMG-måler. Dette afsnit forholder sig ikke til design disse kommercielle komponenter, men om de kan leve op til de krav som er nødvendig for at realisere det ønskede produkt. Desuden indeholder bilaget et sekvens diagram, der  giver en detaljeret beskrivelse af udvalgte funktioner/metoder, som styrer synkerefleksmonitorens hardware del.   

\begin{figure}[H]
\centering
{\includegraphics[width=\linewidth]
{Figure/Blokaede}}
\caption{Figuren viser de enkelte komponenter, som skal designes for at realisere synkerefleksmonitoren.}
\label{blokflow}
\end{figure}


\chapter{Hardware}

\subsection{Analog Discovery}
Analog Discovery (AD) bruges i dette projekt til to formål. Det første er at AD skal fungere som funktionsgenerator og det andet er at den den skal også fungere som dataopsamlingmodul. Figur \ref{ADogINA128} viser et skitse af de to formål som AD bliver brugt til. Her ses det at AD generere et AC signal på 2V, som sendes til indgangen af instrumentationsforstærkeren INA128. Dette signal bliver brugt til at generere en konstant strøm ud af operationsforstærkeren LM318. Det ses også på figuren at AD modtager et signal fra et antialiaseringsfilter. Her fungerer AD en dataopsamlingsmodul, der konverter et analog signal til et digital signal. 


   
\begin{figure}[H]
\centering
{\includegraphics[width=10cm]
{Figure/ADogINA128}}
\caption{Figuren viser ADs funktion i det samlede system. AD fungerer som funktionsgenerator og som dataopsamlingsmodul}
\label{analogOgINA}
\end{figure}


For at rekonstruere et fysiologisk signal som eksisterer i et analog domæne til et digital domæne kræver det at man opfylder en række krav. Den første krav er at overholde Shannons samplingsteorem som siger at samplingsfrekvensen skal minimum være 2 gange den maksimale frekvenskomponent, Nyquist-frekvensen. I praksis ønsker man en måling, der ikke indeholder frekvenser, som er større end den halve samplingsfrekvens dvs. Nyquist-frekvensen. Frekvenser større end Nyquist-frekvensen giver anledning til aliasering, der gør det vanskeligt at rekonstruere det oprindelige signal korrekt. For at genskabe et signal bedst muligt anbefales det at man vælger en samplingsfrekvens, der er betydelig højere end den maksimale frekvenskomponent. Valg af dataopsamlingsenhed har også betydning for hvor præcise man kan genskabe et analog signal. Konvertering af analoge værdier til digitale værdier sker ved at dataopsamlingsenheden måler et analog signal værdi, som derefter konverteres til den nærmeste digital værdi. Under konvertering opstår der fejl, da konverteringen ikke er præcis, men approksimation. Denne fejl kaldes kvantiseringsfejl.  Konverteringens præcision bestemmes af ADC'ens spændingsområde og opløsning. Forholdet mellem ADC'ens inputområde og dens kvantificeringsniveauer/ADC’ens opløsning kan udtrykkes ved denne formel  \cite [s. 634-635]{Lyons}:

\begin{equation}
\label{eq2.1}
 LSB=  \dfrac{{Spændingsområde}}{2^{bits}} 
\end{equation}

Denne formel udtrykker A/D konverterens mindste detekterbare spændingsændring, Least Significant Bit. I dette projekt benyttes AD, der kan yde 14-bit analog til digital-konvertering og som får en spændingsforsyning på 8 V. Med disse opløsninger kan LSB beregnes som følgende:

 \begin{equation}
\label{eq2.2}
 LSB=  \dfrac{{8V}}{2^{14}}=0,48 mV 
\end{equation}
 
 Dette betyder at det mindste detekterbare spændingsændring som AD kan detektere er 0.48mV. Spændingsændringer over 0,48mV bliver omsæt til digitale værdier og værdier under den bliver ikke omsæt. AD er valgt ud fra følgende fordele:
 
 \begin{itemize}
\item 	Den kan fungere som funktionsgenerator samtidig med at den læser to signaler ind simultant
\item Den kan sample to signaler simultant med en $f_{s}=500kHz $
\item Den kan fungere som dataopsamlingsenhed 
 
\end{itemize}
Med disse fordele er det besluttet at benytte AD. 
 
 \pagebreak

\subsection{Instrumentationsforstærker}
Når man måler fysiologiske signaler har man bruge for at forstærke dem, da deres amplituder er små. Disse amplituder skal forstærkes fra millivolt  til volt området. Ydermere er disse signaler overlejret med brum støj på 50 Hz, som kommer af de omkringliggende apparater, ved at måleudstyret og måleobjektet har elektromagnetisk kobling imellem hinanden. For at eliminere eller undertrykke denne støj mest muligt anvendes en differensforstærker. Udover at undertrykke denne støj, ønsker man også som omtalt at forstærke indgangssignalet fra funktionsgeneratoren. Derfor anvendes i dette projekt to instrumentationsforstærker af typen INA128P. Med INA128P er det muligt at forstærke et signal ved kun at regulere én modstand. Dette betyder at et ønsket forstærkning kan opnås ved at kun justere den eksterne modstand kaldet $(R_G)$. INA128P har følgende egenskaber som er ønskede, når man måler elektrofysiologiske signaler \cite{PeterJohansen2014}. 

\begin{itemize}
\item 	Høj indgangsimpedans på ca. $10^{10} \Omega $
\item	Stor common mode rejection (CMR) på minimum 120dB
\item 	Differentielt input-single ended out (nødvendigt for at mindske $CM_{noise}$)
\end{itemize}

I dette projekt implementeres to instrumentationsforstærker af typen INA128. Den ene bruges til at forstærke signalet fra funktionsgeneratoren og fjerne brum støj på 50 Hz, og den anden anvendes til både at forstærke elektrofysiologiske signaler fra måleobjektet og undertrykkelse af commen mode støj.


\subsubsection{Instrumentationsforstærker  1}

Da INA128 som nævnte også benyttes til at forstærke signalet fra funktionsgeneratoren på 2V, skal den eksterne modstand, $R_G$, også beregnes for denne. Hvis gain vælges til 2 skal man ifølge databladet  sætte $R_G=50k\Omega$  \citep [s.13]{TexasInstruments2005}. 
Udgangssignalet for denne instrumentationsforstærker beregnes således: 

\begin{equation}
\label{voutina1281}
V_{in} = 2 $$ $$
Gain = 2 $$ $$
V_{outINA128}=2V \times 2=4V
\end{equation}

Med en kendt Gain kan der nu beregnes $R_G$:


\begin{equation}
\label{eq2.7}
Gain = 2 $$ $$ \\
Gain  =2 + \dfrac{{50k\Omega}}{R_G} \Rightarrow {R_G}=50k \Omega
\end{equation}

\begin{figure}[H]
\centering
{\includegraphics[width=10cm]
{Figure/instrumentation1}}
\caption{Figuren viser diagrammet over instrumentationsforstærker 1. Den bruges til undertrykkelse af støj og forstærkning af spænding fra Analog Discovery.}
\label{fig:instrumentation1}
\end{figure}

\begin{table}[H]
\centering
\caption{Stykliste for instrumentationsforstærker 1}
\label{Styklisteinstrumentationsforstarker1}
\begin{tabular}{|l|l|}
\hline
\multicolumn{2}{|c|}{\textbf{Stykliste}} \\ \hline
INA128             &                    \\ \hline
$R_{G}$            & 51 kohm            \\ \hline
\end{tabular}
\end{table}



\subsubsection{Instrumentationsforstærker  2}

Instrumentationsforstærker 2 anvendes til både at forstærke elektrofysiologiske signaler fra måleobjektet og undertrykkelse af commen mode støj. 


Ved anvendelse af INA128 kan man forstærke et signal 100 gange og stadig have en tilstrækkelig båndbredde, der ligger over anti-aliaseringsfilterets knækfrekvens på $25 kHz$. Dette kan aflæses på figur \ref{Fig:GainOgfrequnecy}. 

\begin{figure}[H]
\centering
{\includegraphics[width=10cm]
{Figure/GainOgfrequnecy}}
\caption{Figuren viser at Gain på 10 V/V bliver ved at være konstant op til ca. 500 kHz. Vælger man derimod gain på 100 V/V så er gain konstant kun op til ca. 100 kHz \cite{TexasInstruments2005} }
\label{Fig:GainOgfrequnecy}
\end{figure} 

Det ses på figur \ref{Fig:GainOgfrequnecy} at båndbredden BW er større end anti-aliaseringsfilterets knækfrekvens på 20khz, når gain vælges til 100. Dette betyder at forstærkeren er bred nok til at kunne indeholde frekvenser, der er større og mindre end knækfrekvensen.  Med denne aflæsning af BW er det sikrede at INA128 kan benyttes til formålet. 

Den ønskede forstærkning reguleres vha. $R_G$ og det kan udledes af denne formel:

\begin{equation}
\label{eq2.3}
Gain  =1 + \dfrac{{50k\Omega}}{R_G}
\end{equation}

Ved at sætte Gain = 100 kan man beregne $ R_{G}$:

\begin{equation}
\label{eq2.4}
Gain = 100 $$ $$ \\
Gain  =100 + \dfrac{{50k\Omega}}{R_G} \Rightarrow {R_G}=505 \Omega
\end{equation}

For at opnå gain modstanden, sættes en modstand på 470ohm sammen med en modstand på 36ohm i serie.

\begin{figure}[H]
\centering
{\includegraphics[width=10cm]
{Figure/instrumentation2}}
\caption{Figuren viser diagrammet over instrumentationsforstærker 2. Den bruges til undertrykkelse af støj og forstærkning af biologiske signaler fra måleobjekt.}
\label{Fig:instrumentation2}
\end{figure} 



\begin{table}[H]
\centering
\caption{Stykliste for instrumentationsforstærker 2}
\label{Styklisteinstrumentationsforstarker2}
\begin{tabular}{|l|l|}
\hline
\multicolumn{2}{|c|}{\textbf{Stykliste}} \\ \hline
INA128             &                    \\ \hline
$R_{G}$            & 470ohm+36ohm            \\ \hline
\end{tabular}
\end{table}


For at teste om INA128 kan i praksis levere en gain på 100 laves der et spændingsdeler kredsløb, der modtager et signal på 1V, som derefter bliver formindsket til $10mV$. Dette spændingsdeler kredsløb er nødvendig pga:

\begin{itemize}
\item 	at man ikke direkte kan sende små signaler fra AD til INA128 pga. AD har store usikkerheder, når den genererer små signaler 
\item	at man ikke kan sende 1V igennem INA128 med gain på 100. Dette giver en udgangsspænding på 100V, som INA128 ikke kan klare. 

\end{itemize}

Med de to begrundelse designes følgende spændingsdeler kredsløb:


\begin{figure}[H]
\centering
{\includegraphics[width=10cm]
{Figure/spaedingdeler}}
\caption{Figuren viser kredsløbet til spændingsdeleren}
\label{Fig:spaedingdeler}
\end{figure} 

\begin{equation}
\label{eq2.5}
V_{inINA128} = 10 \times 10^{-3}V $$ $$ \\
R_{1} = 100k\Omega, R_{2} =1000\Omega $$ $$ \\
V_{outIna128}  = 100 \times \dfrac{R_{2}}{R_{1}+ R_{2}}=1V
\end{equation}

Det forventes at INA128 forstærker 10mV til 1V, hvilken svar til en gain på 100.  

\begin{table}[H]
\centering
\caption{Stykliste for spændingsdeleren}
\label{Styklistespandingsdeleren}
\begin{tabular}{|l|l|}
\hline
\multicolumn{2}{|c|}{\textbf{Stykliste}} \\ \hline
R1             &      100kohm              \\ \hline
R2             &         1kohm  			\\ \hline
R3             &           1kohm         \\ \hline
R4             &         100kohm  			\\ \hline


\end{tabular}
\end{table}



\subsection{Strømgenerator}

Da bioimpedans måling kræver at man sender en konstant strøm til måleobjektets væv er det nødvendigt at designe og opbygge en strømgenerator, der kan levere en konstant strøm. Til formålet anvendes operationsforstærkeren LM318 og for at sikre at LM318 kan bruges udregnes slewrate\cite{Baker2003}. $$slewrate=2*\pi*f*V$$
$$2*\pi*20kHz*4V=0,503V/us$$
Ved at kigge i databladet for LM318 er der oplyst en typisk slewrate på 70V/us, hvilke er fint til den beregnet  på 0,503V/us.

LM318 indeholder én operationsforstærker som kan bruges til at designe en basic Howland.

Derfor vælges LM318 til at levere det ønsket strøm. Det forventet strømoutput som LM318 leverer til vævet, når $R_4 =10k\Omega $, kan beregnes som følgende:





\begin{equation}
\label{eq2.8}
I_{\textbf{væv}} = 2 \times \dfrac{V_{in}}{R_{5}}  $$ $$
I_{\textbf{væv(p - p)}} = 2 \times \dfrac{4_{vp}}{10k}=800uA
\end{equation}
Dette resultat omregnes til rms:
 
\begin{equation}
\label{eq2.8}
I_{\textbf{(peak)}} = \dfrac{I_{\textbf{væv(p - p)}}}{2}=400uA
$$ $$
I_{\textbf{rms}} = \dfrac{I_{\textbf{(peak)}}}{\sqrt{2}}=283uA
\end{equation}

Denne konstant strøm sendes til måleobjektets væv. 

\begin{figure}[H]
\centering
{\includegraphics[width=10cm]
{Figure/StromgeneratorLM318.PNG}}
\caption{Figuren viser kredsløbet til strømgeneratoren LM318}
\label{Fig:StromgeneratorLM318}
\end{figure} 

\begin{table}[H]
\centering
\caption{Stykliste for strømgenerator}
\label{Styklistestromgenerator}
\begin{tabular}{|l|l|}
\hline
\multicolumn{2}{|c|}{\textbf{Stykliste}} \\ \hline
LM318N             &                    \\ \hline
R1             &         10kohm  			\\ \hline
R2             &           10kohm         \\ \hline
R3             &         10kohm  			\\ \hline
R3             &         10kohm  			\\ \hline

\end{tabular}
\end{table}


\subsection{OP-AMP}
Outputtet af instrumentationsforstærker 2 skal yderligere forstærkes og dette sker ved at benytte en ikke-inverterende operationsforstærker som ser således ud: 

\begin{figure}[H]
\centering
{\includegraphics[width=10cm]
{Figure/IkkeInviterendeOpAmp}}
\caption{Figuren viser kredsløbet til den ikke-inverterende operationsforstærker LM318N.}
\label{Fig:IkkeInviterendeOpAmp}
\end{figure} 




Det er nødvendigt at forstærke signalet til denne størrelse, da man vil udnytte A/D konverterens inputområde, som ligger mellem $\pm 25$ \citep{NI}. Man kunne også vælge at udnytte hele A/D konverterens dynamikområde, men det er valgt at give A/D konverteren en buffer for at imødekomme signaler, der har en markant afvigelsesprocent. Hvis disse type signaler forekommer og man ikke giver A/D konverteren en buffer, så mister man al data, som overskrider de 10V.   
Der benyttes operationsforstærkeren LM318 til at forstærke signalet op, hvor Gain er bestemt af forholdet mellem to modstande, $R_1$  og $R_2$: 

\begin{equation}
\label{eq2.10}
 Gain= 1 + \dfrac{R_{2}}{R_{1}} $$ $$
\end{equation}

For at forstærke signalet op til $10V$ kræver det at man gainer $V_{out}$ 10 gange og vælger $R_1=1k\Omega$. 
Hermed isoleres værdien af $R_2$:

\begin{equation}
\label{eq2.11}
 10= 1 + \dfrac{R_{2}}{1k\Omega} \Rightarrow R_{2} =9k\Omega
\end{equation}
Det forventes at operationsforstærkeren forstærker $V_{out}$ til $10V$.\

Slew rate udregnes for at sikre at de 10 V kan håndteres af LM318.
$$slewrate=2*\pi*f*V$$
$$2*\pi*20kHz*10V=1,257V/us$$
Ved at kigge i databladet for LM318 er der oplyst en typisk slewrate på 70V/us, hvilke er fint til den beregnet  på 1,257V/us.
\begin{table}[H]
\centering
\caption{Stykliste for OP-AMP}
\label{Styklisteopamp}
\begin{tabular}{|l|l|}
\hline
\multicolumn{2}{|c|}{\textbf{Stykliste}} \\ \hline
LM318N             &                    \\ \hline
R1             &         1kohm  			\\ \hline
R2             &           9,1kohm         \\ \hline

\end{tabular}
\end{table}



\subsection{AA filter}

I dette projekt vælges samplingfrekvensen til 500 kHz. Den halve samplingfrekvens  bliver derfor 250 kHz (Nyquist frekvensen). Da Analog Discovery bruges som A/D konverter, skal signalet ved 250 kHz være dæmpet under $1/2*LSB$, som i dB svarer til en dæmpning på $20log*2^{15}= 90 dB$ 

På baggrund af modultesten i \textit{Bilag 6 - Implementering \& Test} kan det aflæses at signalet i sig selv allerede er dæmpet med ca. 75 dB, derfor skal der som minimum tilføres en yderligere dæmpning på 20 dB.

Derfor skal der designes et anti-aliaseringsfilter som tillader passering af frekvenser, der er mindre end Nyquist frekvensen og dæmper frekvenser, som er højere end Nyquist frekvensen. Dette anti-aliaseringsfilter skal give en minimum dæmpning på 20 dB ved 250 kHz. Dette kan lade sig gøre med et første ordens filter, da det dæmper med 20 dB pr. dekade. Der vælges i stedet for et 2. ordens filter, der vil dæmpe 40 dB pr. dekade, for at tage højde for variationer i det optagede signal.

Samtidig er der taget højde for det amplitude moduleret signal som stadig ligger indenfor passbåndet ved 20 kHz, således at synkesignalet ikke bliver filteret væk.

Lavpas filteret designes med følgende specifikationer:
\begin{itemize}
\item Aktivt lavpas
\item 2. orden Butterworth 
\item Sallen key
\item Knækfrekvens = 25 kHz
\end{itemize}

De overstående specifikationer er indtastet i programmet FilterPro og der er beregnet følgende kredsløb i figur \ref{Fig:aafilterdiagram}. Som operationsforstærker vælges OP27G og for at sikre at OP27G kan bruges udregnes slewrate:
$$slewrate=2*\pi*f*V$$
$$2*\pi*20kHz*10V=1,257V/us$$

I OP27G's datablad kan slewrate aflæses til 2,8 V/us, hvilket er fint for den udregnet slewrate på 1,005 V/us. 

Derudover udregnes Gain Bandwidth Product (GBWP)\cite{Baker2003}, som skal være lig med eller større end $100*fc$. Det udregnes, med en knækfrekvens på 25 kHz, til at blive $100*25000=2,5MHz$. I OP27G's datablad kan GBWP aflæses til 8 MHz, hvilket stemmer fint med den udregnet på 2,5 MHz.



\begin{figure}[H]
\centering
{\includegraphics[width=10cm]
{Figure/aafilterdiagram}}
\caption{Figuren viser kredsløbet til AA filteret.}
\label{Fig:aafilterdiagram}
\end{figure} 

Bodeplottet viser at AA filteret dæmper med 40 dB ved 250 kHz som ønsket. Udover kan det aflæses at ved knækfrekvensen på 25 kHz, at der er en korrekt faseforskydning på \ang{90}.

\begin{figure}[H]
\centering
{\includegraphics[width=\linewidth]
{Figure/aafilterbode}}
\caption{Bodeplot for filterets dæmpning. }
\label{Fig:aafilterbode}
\end{figure} 


\begin{table}[H]
\centering
\caption{Stykliste for AA filter}
\label{Styklisteaafilter}
\begin{tabular}{|l|l|}
\hline
\multicolumn{2}{|c|}{\textbf{Stykliste}} \\ \hline
OP27G             &                    \\ \hline
R1             &         3kohm  			\\ \hline
R2             &           5,6kohm         \\ \hline
C1             &           1nF         \\ \hline
C2             &           2.2nF         \\ \hline

\end{tabular}
\end{table}

%På baggrund af modultesten i \textit{Bilag 7 - Implementering \& Test} skal filteret i dette projekt realiseres som et 2. ordens butterworth lavpasfilter af typen Sallen-Key, da det har et flat filterrespons i passbåndet,  med cutoff frekvens på 25 kHz. Filteret har nedenstående overføringsfunktion: 
% \\linebreak \textbf{Indsæt et billede af filtret man nu vælger}
%
%\begin{equation}
%\label{eq2.14}
% H(s)=  \dfrac{2 \times \pi \times f_{c}}{s^{2}+ 2 \times \varsigma \times (2 \times \pi \times f_{c}) s + (2 \times \pi \times f_{c})^{2} } = 0,12mV
%\end{equation} 
%
%Filtreret knækfrekvens kan bestemmes vha. denne formel: 
%\begin{equation}
%\label{eq2.15}
%f_{c}= \dfrac{1}{2 \times \pi \times  \sqrt{R1 \times C1 \times R2 \times C2}}
%\end{equation}
%
%I stedet for at beregne værdierne for komponenterne, anvendes et værkstøj til design af filtre, som kan regne disse komponentværdier når man indtaster den ønskede knækfrekvens 5 . Indtastning af cutoff frekvensen giver følgende resultater:
%
%\begin{equation}
%\label{eq2.16}
% R1=R2= 33k\Omega $$ $$
% C1=C2=100pF
%\end{equation}
%
%\textbf{Indsæt podeplot}
%\textbf{Der mangler to punkter:
%	Hvor meget er signalet dæmpede allerede?
%	Hvor skal vi dæmpe det yderligere for komme ned til 100db? Vi kender vel først den faktiske dæmpning der er behov for ved integrationstesten?} 
	
	
	
\chapter{Software}
I forlængelse af software arkitekturen i \textit{"bilag 4 - Arkitektur"}, hvor grænsefladerne til hver enkelt software blok blev beskrevet,  praktiseres nu et software design. Denne designprocess er til for at det bliver klart hvor den tilhørende Matlab kode kan skrives i den senere implementering af Synkerefleksmonitor.

Først er der detaljer om væsentlige Matlab funktioner i form af beskrivelser af funktionernes indhold opsat i tabelform. Dernæst følger funktionernes input og output parametre i tabelform.

Der er yderligere beskrevet indholdet af funktionen \textit{"Process\_Measurements"} vha. UML-aktivitetsdiagram. Efterfølgende er den overordnet rækkefølge af programmets kodeeksekvering udført i et sekvens diagram.

Til slut vises et design af Matlab GUI med tilhørende objekter, såsom knapper, tekst og graf, hvilke sundhedspersonalet kan interagere med. 


\subsection{Funktioner}
\begin{table}[H]

\begin{tabularx}{\textwidth}{l l X}
     Navn	&	Type		&	Beskrivelse \\ \midrule
     Synkerefleksmonitor\_OpeningFcn   	& 	Callback  	& 	Denne funktion er automatisk oprettet af Matlab og køre så snart programmet startes. Den indeholder funktionen "Start\_GUI".  \\ 			  \addlinespace[2mm]
     Btn\_Start\_Measurements	&	Callback	&	Denne funktion bliver kørt, når der trykkes på knappen "Start Measurements". De funktioner som efterfølgende bliver kørt er: "Generate\_SineWave", "Read\_Measurments", "Process\_Measurments" og "Show\_Measurments". \\   \addlinespace[2mm]
     Btn\_Save\_Measurements	&	Callback	&	Denne funktion bliver kørt, når der trykkes på knappen "Save Measurements". Den indeholder funktionen Save\_Measurements.\\   \addlinespace[2mm]
     Start\_GUI	&	Funktion	&	Denne funktion indeholder kode som automatisk bliver kørt når GUI'en åbnes. Koden indeholder visning af billede, dato og tid, samt deaktivering knappen \textit{"Save Measurements"}. \\   \addlinespace[2mm]
     Generate\_SineWave	&	Funktion	&	Funktionen opretter forbindelse til Analog Discovery, tilføjer funktionsgeneratoren samt dens indstillinger, amplitude 2 V og frekvens 20 kHz.\\   \addlinespace[2mm]
     Read\_Measurements	&	Funktion	&	Funktionen tilføjder de to analog input som bruges til måling af BI og EMG. Yderligere sættes samplingfrekvensen til 500 kHz og at en måling skal vare 10 sekunder.\\   \addlinespace[2mm]
     Process\_Measurements	&	Funktion	&	Her bliver der foretaget envelope af BI målingen. Dette sker ved at dobbeltensrette BI-signalet og lavpasfiltrere det ved 500 Hz.   \\   \addlinespace[2mm]
     Save\_Measurements	&	Funktion	&	Målingerne BI og EMG, gemmes som en samlet CSV-fil.\\   \addlinespace[2mm]
     Show\_Measurements	&	Funktion	&	Til slut vises det udglattet BI og EMG. Der opsættes akser, labels og title på grafen.\\   \addlinespace[2mm]
     Load\_Measurements	&	Funktion	&	Henter tidligere gemte målinger klar til visning.\\   \addlinespace[2mm]
   
     \bottomrule                                                                                                                   
    \end{tabularx}
    \caption {Tabellen viser beskrivelse om funktionerne der er designet for Synkerefleksmonitor software del.}
    \label{tab:sw1}
	
\end{table}



\begin{table}[H]
\center
\begin{tabularx}{\linewidth}{l  X  X}
     \textbf{Funktion}	&	\textbf{Input}		&	\textbf{Output} \\ \midrule
     
     Start\_GUI	&		&	Set Date(now)\\   \addlinespace[2mm]
     			&		&	Set Picture\\   \addlinespace[2mm]\addlinespace[2mm]\hline\addlinespace[2mm]
  
	Generate\_SineWave   	&		&	handles.GS\\   \addlinespace[2mm]\hline\addlinespace[2mm]
Read\_Measurements	&	handles.GS	&	handles.BI\\   \addlinespace[2mm]
			    &		&	handles.EMG\\   \addlinespace[2mm]
			    &		&	handles.timestamps\\   \addlinespace[2mm]\hline\addlinespace[2mm]
			    Process\_Measurements    &	handles.BI	&	handles.locs\_syn\\   \addlinespace[2mm]
						&	handles.EMG	&	handles.BIsignal\\   \addlinespace[2mm]
						&	handles.timestamps	&	handles.EMGsignal\\   \addlinespace[2mm]
						&	& handles.TID	\\   \addlinespace[2mm]\hline\addlinespace[2mm]
						
				Save\_Measurements    &handles.BI	&	CSV-fil\\   				\addlinespace[2mm]
			    &	handles.EMG	&	\\   \addlinespace[2mm]
			    &	handles.timestamps	&	\\   \addlinespace[2mm]\hline\addlinespace[2mm]
   		
Show\_Measurements	    &	handles.TID&	Subplot measurements\\   \addlinespace[2mm]
   		    			&	handles.locs\_synk&						\\   \addlinespace[2mm]
						&	handles.BIsignal&						\\   \addlinespace[2mm]
   		    			&	handles.EMGsignal&						\\   \addlinespace[2mm]\hline\addlinespace[2mm]   
   		Load\_Measurements	    &	CSV-fil	&	handles.BI\\   \addlinespace[2mm]
   		&		&	handles.EMG\\   \addlinespace[2mm]
   		&		&	handles.timestamps\\   \addlinespace[2mm]
     \bottomrule                                                                                                                   
    \end{tabularx}
    \caption {Tabellen viser funktionernes vigtigste input og output parametre.}
    \label{tab:sw1}
	
\end{table}




\begin{figure}[H]
\centering
{\includegraphics[width=8cm]
{Figure/designUML}}
\caption{Figuren viser indholdet af kodeeksekvering i funktionen \textit{"Process\_Measurements"} i den del hvor der udføres envelope af det rå BI-signal. Først bliver det rå BI signal dobbeltensrettet og dernæst filtreret vha. digital lavpasfilter med knækfrekvens = 500 Hz og dæmper med 40 dB over en dekade. Til sidst bliver signalet detrended, downsampled og synk detekteres.}
\label{Fig:designUML}
\end{figure} 








\subsection{Sekvens Diagram}

Sekvensdiagrammet viser rækkefølgen af programmets kodeeksekvering. Programmet igangsættes af brugeren ved ved at åbne m.filen Synkerefleksmonitor. Her ligger GUI'en til hele programmet. Brugeren kan herfra vælge at igangsætte en måling ved at trykke knappen \textit{"Btn\_Start\_Measurements"}. Denne knap kører efterfølgende en række funktioner, der tilsammen måler og viser en bioimpedans og EMG måling simultant. Herefter er det muligt at gemme den foretaget måling ved knappen \textit{"Btn\_Save\_Measurements"} i en csv-fil. Gemte målinger kan hentes frem ved knappen \textit{"Btn\_Load\_Measurements"}.
De selvudviklede funktioner eksisterer i egne m-filer. Denne opdeling muliggør at man nemt kan udvide programmet, uden at have kendskab til alle disse funktioner.    

\begin{figure}[H]
\centering
{\includegraphics[width=\linewidth]
{Figure/SekevensDiagram}}
\caption{Figuren viser sekvensen af programmets kode}
\label{Fig:SekevensDiagram}
\end{figure} 



\subsection{GUI}

Synkerefleksmonitor består af en Graphical User Interfaces (GUI), hvor GUI'en interagere med sundhedspersonalet. Når programmet startes op vil man få vist følgende vindue, se figur \ref{Fig:designGUI}. Denne GUI består af forskellige objekter. I GUI'en er det muligt at se dags dato og klokken for målingen. Sundhedspersonalet har mulighed for at trykke på tre knapper. 

\begin{figure}[H]
\centering
{\includegraphics[width=\linewidth]
{Figure/designGUI}}
\caption{Figuren viser designet af GUI til Synkerefleksmonitor.}
\label{Fig:designGUI}
\end{figure} 

Knappen \textit{"Btn\_Start\_Measurements"}, bruges når der ønskes en måling. Knappen \textit{"Btn\_Save\_Measurements"}, når målingen skal gemmes og knappen \textit{"Btn\_Load\_Measurements"}, som henter en tidligere måling.

Brugeren bliver informeret om hvad SRM foretager sig, da nogle af funktionerne kan tage længere tid. Det er beskeder om at der måles, gemmes eller hentes en måling. Dette sker ved en gul besked henover GUI, se figur \ref{Fig:designmessageboks1}.
 
 \begin{figure}[H]
\centering
{\includegraphics[width=10cm]
{Figure/designmessageboks1}}
\caption{De tre beskeder som kommer frem når der måles, gammes eller hentes.}
\label{Fig:designmessageboks1}
\end{figure} 
 
 
Når der gemmes en måling kommer yderligere en massages boks frem, figur \ref{Fig:designmessageboks3} for at informere om at målingerne nu er gemt i csv-fil.

\begin{figure}[H]
\centering
{\includegraphics[width=4cm]
{Figure/designmessageboks3}}
\caption{Massages boksen der bekræfter at målingerne er gemt.}
\label{Fig:designmessageboks3}
\end{figure} 

GUI'en består af en rækker objekter som sundhedspersonalet kan interagere med i form af knapper, tekstbokse og en drop down menu, se figur \ref{Fig:designGuikontrol}. Udover at starte, gemme og hente en måling ved brug af knapperne, er det muligt at få vist dato og tid men også hvor mange synk der er 
detekteret i målingen. Det er også muligt at sætte målingens tid fra 2 til 12 sek. Til sidst er det muligt at indtaste strømmen, hvis denne har ændret sig. Undervejs i programmet ,hvor det er hensigtsmæssigt, vil nogle af knapperne blive deaktiveret for at undgå fejlmålinger. 


\begin{figure}[H]
\centering
{\includegraphics[width=4cm]
{Figure/designGuikontrol}}
\caption{Kontrolpanelets funktioner}
\label{Fig:designGuikontrol}
\end{figure} 











\bibliography{library}
\end{document}


