\input{setup}
\begin{document}
\begin{titlingpage}
\begin{center}

~ \\[3cm]

%\includegraphics[width=0.6\textwidth]{figurer/ASE}~\\[1cm]
\textsc{\LARGE Bilag 15}\\[1.5cm]

%\textsc{\Large Sundhedsteknologi}\\
%\textsc{\Large 3. semesterprojekt}\\[0.5cm]

\noindent\makebox[\linewidth]{\rule{\textwidth}{0.4pt}}\\
[0.5cm]{\Huge Værktøjer}
\noindent\makebox[\linewidth]{\rule{\textwidth}{0.4pt}}
\end{center}
\vfill
\begin{center}
{\large 19. december 2017}
\end{center}
\end{titlingpage}

\newpage
\tableofcontents*
\newpage

\chapter{Værktøjer}
De nedenstående værktøjer er anvendt igennem udviklingsprocessen af Synkerefleksmonitor. 

\section{Overleaf} 
Programmet Overleaf er et online værktøj, der giver mulighed for at skrive dokumenter i latex. Forprojektet og det meste af bachelor opgaven er skrvet i Overleaf, men på grund af opbrugt af Overleaf blev det hele eksporteret ud af Overleaf igen og flyttet over i Github.

\section{Google drive/docs/mail}
Google drive blev anvendt som lager til artikler, filer, datablade.
I google docs blev der ført møder og logbog. Der blev oprettet en gmail konto til mail korrespondance med vejleder og eksterne.

\section{Facebook}
Den daglige kontakt i gruppen blev ført på Facebook i en privat gruppe samt i deres messenger.

\section{Onenote}
Ved litteratursøgning blev Onenote brugt som noter, som blev delt i gruppen.

%\section{GradePro (gradering af artikler)}
\section{Teamgantt}
I online portalen teamgannt blev der oprettet en tidsplan, efter ASE-modellen, som blev løbende opdateret. Alle gruppemedlemmer og vejleder havde adgang til.

\begin{figure}[H]
\centering
{\includegraphics[width=\linewidth]
{Figure/teamgantt}}
\caption{Interfacet hos teamgantt.}
\label{teamgantt}
\end{figure}

\section{Pivotal tracker}
Pivotal blev brugt til at oprettet mindre opgaver løbende i bachelorprojektet. Pivotal kørte efter Scrum princippet hvor den var opsat til ugentlige sprint. Hvert gruppemedlem havde mulighed for at oprettet opgaver til både sig selv og andre i gruppen. Prioriteringen skulle sættes hver gang en opgave oprettes. Der skulle hele gruppen være enig om denne valgt prioritering. 
\begin{figure}[H]
\centering
{\includegraphics[width=\linewidth]
{Figure/pivotal}}
\caption{Interfacet hos pivotal.}
\label{pivotal}
\end{figure}

\section{Mendeley}
Programmet Mendeley blev brugt til at oprette og vedligeholde en database for gruppens samlet litteratur. Mendeley oprettede selv et bibliotek, som kunne importes til latex. På denne måde var det overskueligt og henvisninger var nemme at udføre.

\begin{figure}[H]
\centering
{\includegraphics[width=\linewidth]
{Figure/mendaley}}
\caption{Interfacet hos mendaley.}
\label{mendaley}
\end{figure}


\section{Multisim 14.0}
Multisim gav mulighed for at løbende simulere kredsløb igennem udviklingsprocessen, inden de blev bygget på fumlebræt. Dog var det ikke alle komponenter som findes i Multisim og var nødsaget til at gå direkte til fumlebræt.

\begin{figure}[H]
\centering
{\includegraphics[width=\linewidth]
{Figure/multisim}}
\caption{Interfacet hos multisim.}
\label{multisim}
\end{figure}



\section{FilterPro}

Programmet FilterPro blev brugt når der skulle realiseres et filter. Ved at indtaste det ønskede filter, beregner FilterPro og kommer med kredsløb design med stykliste. Det er også muligt at ændre værdier og løbende se på bodeplot når man er tilfreds med filteret. 

\begin{figure}[H]
\centering
{\includegraphics[width=\linewidth]
{Figure/filterpro}}
\caption{Interfacet hos FilterPro.}
\label{filterpro}
\end{figure}


\section{WaveForms}
I forbindelse med test var det WaveForms som styrede Analog Discovery. I WaveForms blev der brugt funktionerne scope, wavegen, supplies, logger, spectrum og network.


\begin{figure}[H]
\centering
{\includegraphics[width=\linewidth]
{Figure/waveformsVaerktoj}}
\caption{Interfacet hos WaveForms.}
\label{waveformsVaerktoj}
\end{figure}


\section{Github}

Da der ikke var længere plads på Overleaf, blev latex importeret over i Github. Det var nu muligt at skrive videre i latex uden at løbe ind i pladsproblemer igen. Girhub gav mulighed for at løbende commit det man havde lavet, så der var en backup af det, men også så andre gruppemedlemmer kunne se den nyeste version af bilag og rapport. 

\begin{figure}[H]
\centering
{\includegraphics[width=\linewidth]
{Figure/github}}
\caption{Interfacet hos Github.}
\label{github}
\end{figure}


\section{texmaker}
Texmaker blev valgt da behovet for Github opstod. Texmaker er programmet hvor selve skrivningen foregår i sproget latex. Latex er generelt valgt da det håndtere større rapporter rigtig godt, og så er der allerede et kendskab til det igennem tidligere semestre. 

\section{Visio}
Visio blev brugt til at lave alle diagrammer eller bare hvor der var behov for at lave en figur med flere komponenter. Pakken sysML var installeret hvilet det var muligt at lave BDD,IBD,UML og SD.

%\bibliography{library}
\end{document}


