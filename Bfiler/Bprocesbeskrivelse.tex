%\documentclass[11pt,oneside,a4paper,openright]{report}
%\usepackage[utf8]{inputenc}
%\renewcommand{\contentsname}{Indholdsfortegnelse}
%\usepackage{pdfpages}
%\usepackage{titlesec}
%\titleformat{\chapter}{\normalfont\huge}{\thechapter.}{20pt}{\huge\it}

%%%% Dokumentklassen %%%%

\documentclass[a4paper,11pt,dvipsnames,oneside,openany]{memoir} 	% Openright åbner kapitler på højresider (openany begge)
% fleqn = flush left equation - sikre at alle ligninger tvinges til venstre. I 3. semesterprojektet, skulle ligningerne stå i midten derfor er denne pakke slettet fra dokumentklassen.

\usepackage{subfiles}
\usepackage{nameref}
\usepackage{tabularx}
\usepackage{multirow}
\usepackage[table]{xcolor}


%%%% PACKAGES %%%%

%% Oversættelse og tegnsætning %%
\usepackage[utf8]{inputenc}					% Input-indkodning af tegnsæt (UTF8)
\usepackage[danish]{babel}					% Dokumentets sprog
\usepackage[T1]{fontenc}				    % Output-indkodning af tegnsæt (T1)
\usepackage{ragged2e,anyfontsize}			% Justering af elementer
%\usepackage{fixltx2e}						% Retter forskellige fejl i LaTeX-kernen
\usepackage{titletoc}
\newcommand{\nocontentsline}[3]{}
\newcommand{\tocless}[2]{\bgroup\let\addcontentsline=\nocontentsline#1{#2}\egroup}									% Giver mulighed for at fjerne section nummer i indholdsfortegnelse ved \tocless


\usepackage{lastpage}						% Total antal sider opdateres automatisk ved \pageref{LastPage}
\usepackage{tikz}							% Til at lave flow diagrammer
\usetikzlibrary{calc,trees,positioning,arrows,chains,shapes.geometric,decorations.pathreplacing,decorations.pathmorphing,shapes,matrix,shapes.symbols}				% Til at lave diagrammer
																			
%% Figurer og tabeller (floats) %%
\usepackage{graphicx} 						% Håndtering af eksterne billeder (JPG, PNG, EPS, PDF)
\usepackage{multicol}         	           	% Muliggør output i spalter
\usepackage{rotating}						% Rotation af tekst med \begin{sideways}...\end{sideways}
\usepackage{xcolor}							% Definer farver med \definecolor. Se mere: http://en.wikibooks.org/wiki/LaTeX/Colors
\usepackage{flafter}						% Sørger for at floats ikke optræder i teksten før deres reference
\let\newfloat\relax 						% Justering mellem float-pakken og memoir
\usepackage{float}							% Muliggør eksakt placering af floats, f.eks. \begin{figure}[H]
\usepackage{color, colortbl}				% Tilføjer farve til tabeller

\definecolor{Gray}{gray}{0.9}				% Definerer en farve "yeezy-gray"

%% Matematik mm. %%
\usepackage{amsmath,amssymb,stmaryrd} 		% Avancerede matematik-udvidelser
\usepackage{mathtools}						% Andre matematik- og tegnudvidelser
\usepackage{textcomp}                 		% Symbol-udvidelser (fx promille-tegn med \textperthousand)
\usepackage{rsphrase}						% Kemi-pakke til RS-saetninger, fx \rsphrase{R1}
\usepackage[version=3]{mhchem} 				% Kemi-pakke til flot og let notation af formler, f.eks. \ce{Fe2O3}
\usepackage{siunitx}						% Flot og konsistent præsentation af tal og enheder med \si{enhed} og \SI{tal}{enhed}
\sisetup{output-decimal-marker = {,}}		% Opsætning af \SI (DE for komma som decimalseparator) 

%% Referencer og kilder %%
\usepackage[danish]{varioref}				% Muliggør bl.a. krydshenvisninger med sidetal (\vref)
\usepackage{natbib}							% Udvidelse med naturvidenskabelige citationsmodeller
\usepackage{xr}							    % Referencer til eksternt dokument med \externaldocument{<NAVN>}

%% Misc. %%
\usepackage{listings}						% Placer kildekode i dokumentet med \begin{lstlisting}...\end{lstlisting}
\usepackage{lipsum}							% Dummy text \lipsum[..]
\usepackage[shortlabels]{enumitem}			% Muliggør enkelt konfiguration af lister
\usepackage{pdfpages}						% Gør det muligt at inkludere pdf-dokumenter med kommandoen \includepdf[pages={x-y}]{fil.pdf}	
\pdfoptionpdfminorversion=6					% Muliggør inkludering af pdf-dokumenter, af version 1.6 og højere
\pretolerance=2500 							% Justering af afstand mellem ord (højt tal, mindre orddeling og mere luft mellem ord)


%%%% CUSTOM SETTINGS %%%%

%% Marginer %%
\setlrmarginsandblock{3.0cm}{3.0cm}{*}		% \setlrmarginsandblock{Indbinding}{Kant}{Ratio}
\setulmarginsandblock{3.0cm}{3.0cm}{*}		% \setulmarginsandblock{Top}{Bund}{Ratio}
\checkandfixthelayout 						% Oversætter værdier til brug for andre pakker

%% Afsnitsformatering %%
\setlength{\parindent}{0mm}           		% Størrelse af indryk
\setlength{\parskip}{3mm}          			% Afstand mellem afsnit ved brug af double Enter
\linespread{1,1}							% Linjeafstand

%% Indholdsfortegnelse %%
\setsecnumdepth{subsection}		 			% Dybden af nummererede overskrifter (part/chapter/section/subsection)
\maxsecnumdepth{subsection}					% Dokumentklassens grænse for nummereringsdybde
\settocdepth{subsubsection} 					% Dybden af indholdsfortegnelsen
\setcounter{secnumdepth}{5} 				    % Ekstra subsubsection nummerering
		
%% Opsætning af listings %%
\definecolor{commentGreen}{RGB}{34,139,24}
\definecolor{stringPurple}{RGB}{208,76,239}

\lstset{language=Matlab,				    % Sprog
	basicstyle=\ttfamily\scriptsize,	    % Opsætning af teksten
	keywords={for,if,while,else,elseif,		% Nøgleord at fremhæve
			  end,break,return,case,
			  switch,function},
	keywordstyle=\color{blue},				% Opsætning af nøgleord
	commentstyle=\color{commentGreen},		% Opsætning af kommentarer
	stringstyle=\color{stringPurple},		% Opsætning af strenge
	showstringspaces=false,					% Mellemrum i strenge enten vist eller blanke
	numbers=left, numberstyle=\tiny,		    % Linjenumre
	extendedchars=true, 					    % Tillader specielle karakterer
	columns=flexible,						% Kolonnejustering
	breaklines, breakatwhitespace=true,		% Bryd lange linjer
}

%% Navngivning %%
\addto\captionsdanish{
	\renewcommand\appendixname{Appendiks}
	\renewcommand\contentsname{Indholdsfortegnelse}	
	\renewcommand\appendixpagename{Appendiks}
	\renewcommand\appendixtocname{Appendiks}
	\renewcommand\cftchaptername{\chaptername~}		% Skriver "Kapitel" foran kapitlerne i indholdsfortegnelsen
	\renewcommand\cftappendixname{\appendixname~}	% Skriver "Appendiks" foran appendiks i indholdsfortegnelsen
}

%% Kapiteludssende %%
\definecolor{numbercolor}{gray}{0.7}		            % Definerer en farve til brug til kapiteludseende
\newif\ifchapternonum

\makechapterstyle{jenor}{					        % Definerer kapiteludseende frem til ...
  \renewcommand\beforechapskip{0pt}
  \renewcommand\printchaptername{}
  \renewcommand\printchapternum{}
  \renewcommand\printchapternonum{\chapternonumtrue}
  \renewcommand\chaptitlefont{\fontfamily{pbk}\fontseries{db}\fontshape{n}\fontsize{25}{35}\selectfont\raggedleft}
  \renewcommand\chapnumfont{\fontfamily{pbk}\fontseries{m}\fontshape{n}\fontsize{1in}{0in}\selectfont\color{numbercolor}}
  \renewcommand\printchaptertitle[1]{%
    \noindent
    \ifchapternonum
    \begin{tabularx}{\textwidth}{X}
    {\let\\\newline\chaptitlefont ##1\par} 
    \end{tabularx}
    \par\vskip-2.5mm\hrule
    \else
    \begin{tabularx}{\textwidth}{Xl}
    {\parbox[b]{\linewidth}{\chaptitlefont ##1}} & \raisebox{-15pt}{\chapnumfont \thechapter}
    \end{tabularx}
    \par\vskip2mm\hrule
    \fi
  }
}											        % ... her

\chapterstyle{jenor}						        % Valg af kapiteludseende - Google 'memoir chapter styles' for alternativer

%% Sidehoved %%

\makepagestyle{AAU}							        % Definerer sidehoved og sidefod udseende frem til ...
\makepsmarks{AAU}{%
	\createmark{chapter}{left}{shownumber}{}{. \ }
	\createmark{section}{right}{shownumber}{}{. \ }
	\createplainmark{toc}{both}{\contentsname}
	\createplainmark{lof}{both}{\listfigurename}
	\createplainmark{lot}{both}{\listtablename}
	\createplainmark{bib}{both}{\bibname}
	\createplainmark{index}{both}{\indexname}
	\createplainmark{glossary}{both}{\glossaryname}
}
\nouppercaseheads									% Ingen Caps ønskes

\makeevenhead{AAU}{\small E17BAC-Synk2}{}{\leftmark}	% Definerer lige siders sidehoved (\makeevenhead{Navn}{Venstre}{Center}{Hoejre})
\makeoddhead{AAU}{\rightmark}{}{}		            % Definerer ulige siders sidehoved (\makeoddhead{Navn}{Venstre}{Center}{Højre})
\makeevenfoot{AAU}{\small \thepage \ }{}{ }						% Definerer lige siders sidefod (\makeevenfoot{Navn}{Venstre}{Center}{Højre})
\makeoddfoot{AAU}{}{}{\small \thepage \ }						% Definerer ulige siders sidefod (\makeoddfoot{Navn}{Venstre}{Center}{Højre})

\copypagestyle{AAUchap}{AAU}							% Sidehoved for kapitelsider defineres som standardsider, men med blank sidehoved
\makeoddhead{AAUchap}{}{}{}
\makeevenhead{AAUchap}{}{}{}
\makeheadrule{AAUchap}{\textwidth}{0pt}
\aliaspagestyle{chapter}{AAUchap}					% Den ny style vælges til at gælde for chapters
													% ... her
															
\pagestyle{AAU}										% Valg af sidehoved og sidefod


%%%% CUSTOM COMMANDS %%%%

%% Billede hack %%
\newcommand{\figur}[4]{
		\begin{figure}[H] \centering
			\includegraphics[width=#1\textwidth]{billeder/#2}
			\caption{#3}\label{#4}
		\end{figure} 
}

%% Specielle tegn %%
\newcommand{\decC}{^{\circ}\text{C}}
\newcommand{\dec}{^{\circ}}
\newcommand{\m}{\cdot}


%%%% ORDDELING %%%%

\hyphenation{}


%%%% Tilføjelser af min preample %%%%

% Booktabs:
% The booktabs package is needed for better looking tables. 
\usepackage{booktabs}

% Caption:
% For better looking captions. See caption documentation on how to change the format of the captions.
\usepackage[hang, font={small, it}]{caption}

% Hyperref:
% This package makes all references within your document clickable. By default, these references will become boxed and colored. This is turned back to normal with the \hypersetup command below.
\usepackage{hyperref}
	\hypersetup{colorlinks=false,pdfborder=0 0 0}

% Cleveref:
% This package automatically detects the type of reference (equation, table, etc.) when the \cref{} command is used. It then adds a word in front of the reference, i.e. Fig. in front of a reference to a figure. With the \crefname{}{}{} command, these words may be changed.
\usepackage{cleveref}
	\crefname{equation}{formel}{formler}
	\crefname{figure}{figur}{figurer}	
	\crefname{table}{tabel}{tabeller}

% Mine tilføjelser:
\usepackage{units}                        %% Bruges til at gøre fx 1/2 samlet med: \nicefrac{1}{2}.
\usepackage{tabu, longtable}              %% Bruges til tabeller.
\setlength{\tabulinesep}{1.5ex}           %% Definerer linjeafstand i tabeller.
\usepackage{enumerate}                    %% Bruges til lister.
\usepackage{tabto}                        %% Giver mulighed for TAB med fx \tabto{3em}.
\usepackage[hyphenbreaks]{breakurl}       %% Bruges til websiders url'er.
\renewcommand{\UrlFont}{                  %% Definerer url-font.
\small\ttfamily}                          %
\bibliographystyle{unsrt}                 %% Definere bibliografien. Ses til sidst i dokumentet i kapitlet Litteratur.
\usepackage{amssymb} 
\usepackage{pifont}
%\newcommand{\xmark{\ding{55}}			 % Opretter et unchecked mark

\usepackage[bottom]{footmisc}

\usetikzlibrary{%
    decorations.pathreplacing,%
    decorations.pathmorphing,%
    arrows,
    arrows.meta,
    positioning,
    shapes,
    shadows,
    shapes.geometric
    }
    \usepackage{relsize}

%\definecolor{myblue1}{RGB}{0,157,209}
\definecolor{myblue1}{rgb}{0.12, 0.56, 1.0}
\definecolor{myblue3}{RGB}{216,229,245}
%\definecolor{myblue4}{RGB}{0,149,229}
\definecolor{myblue2}{rgb}{0.19, 0.55, 0.91}
\definecolor{myblue4}{rgb}{0.08, 0.38, 0.74}
\definecolor{myred1}{rgb}{0.82, 0.1, 0.26}
\definecolor{myyellow1}{rgb}{1.0, 0.96, 0.0}
\definecolor{myyellow2}{rgb}{1.0, 0.65, 0.0}


\usepackage{pdflscape}
\usepackage{rotating}

\begin{document}

\begin{titlingpage}
\begin{center}

~ \\[3cm]

%\includegraphics[width=0.6\textwidth]{figurer/ASE}~\\[1cm]

\textsc{\LARGE Bilag 15}\\[1.5cm]

%\textsc{\Large Sundhedsteknologi}\\
%\textsc{\Large 3. semesterprojekt}\\[0.5cm]

\noindent\makebox[\linewidth]{\rule{\textwidth}{0.4pt}}\\
[0.5cm]{\Huge Procesbeskrivelse}
\noindent\makebox[\linewidth]{\rule{\textwidth}{0.4pt}}
\end{center}
\vfill
\begin{center}
{\large 16. december 2017}
\end{center}
\end{titlingpage}

\newpage
\tableofcontents

 \chapter{Forord} 
 \chapter{Indledning}
 I dette bilag er det muligt at læse om bachelorprojektets opbygning og dens gennemførte proces. Bachelorprojektet er opbygget efter bestemte modeller fra læring igennem studietiden. Da dette er et projekt kræver det forberedelse og struktur. Derfor er der beskrevet hele processen om hvordan gruppen er opbygget, gruppens aftaler og samarbejde.

\chapter{Gruppedannelse}
Gruppen består af to gruppemedlemmer. Begge medlemmer er sundhedsteknolgistuderende. Gruppen er dannet på en fælles interesse for projektets beskrivelse og de områder der skal arbejdes med såsom udvikling af software og hardware. Hvor gruppen finder en narturlig deling af disse områder pga. forskellig interesse. 

Gruppens forskellighed er et potentiale for gruppen og de er vigtig at der bliver lyttet til hinandens forslag og idèer. Så gruppen er åbne for andres meninger, men samtidigt ikke være ukritiske. Det er vigtig at man tør sige sine meninger for ellers kan det ske at nogle gode idéer og tanker går tabt.

Der er for begge medlemmer været mulighed for at ytre sine forventninger til projektet og ambitionsniveau. Hvilket har skabt et fælles indtryk af projektet og hvad der forventes. Der er samtidig aftalt hvor vi skal mødes og hvornår på dagen med faste start og slut tidspunkter. Selv der  kun er to gruppemedlemmer er der delt roller ud såsom referent til møder og ordstyrer, som skifter fra gang til gang.

De konkrete aftaler og roller er beskrevet i BilagXX - samarbejskontrakt. 

\chapter{Samarbejdsaftale}
På baggrund af møder i gruppen og med vejleder er der skrevet en samarbejdsaftale. SE BilagXX - Samarbejdsaftale. Her er der beskrevet i bestemte punkter hvordan gruppen skal  arbejde og skal forholde sig under hvert punkt. Denne aftale understøtter således at der ikke opstår misforståelser i gruppen. Den er blevet underskrevet af alle gruppens medlemmer.  


\chapter{Udviklingsforløb}

\tikzset{
>=stealth',
  punktchain/.style={
    rectangle, 
    rounded corners, 
    % fill=black!10,
    draw=black, very thick,
    text width=10em, 
    minimum height=3em, 
    text centered, 
    on chain},
  line/.style={draw, thick, <-},
  element/.style={
    tape,
    top color=white,
    bottom color=blue!50!black!60!,
    minimum width=8em,
    draw=blue!40!black!90, very thick,
    text width=10em, 
    minimum height=3.5em, 
    text centered, 
    on chain},
  every join/.style={->, thick,shorten >=1pt},
  decoration={brace},
  tuborg/.style={decorate},
  tubnode/.style={midway, right=2pt},
}

\begin{tikzpicture}
  [node distance=.8cm,
  start chain=going below,]
     \node[punktchain, join] (projektb) {Projektbeskrivelse};
     \node[punktchain, join] (moscow)      {MoSCoW analyse};
     \node[punktchain, join] (litteratur)      {Litteratursøgning};
     \node[punktchain, join] (krav) {Kravspecifikation};
     \node[punktchain, join] (emperi) {Accepttestspecifikation};
      \node (asym) [punktchain ]  {Asymmetrisk information};
      \begin{scope}[start branch=venstre,
        %We need to redefine the join-style to have the -> turn out right
        every join/.style={->, thick, shorten <=1pt}, ]
        \node[punktchain, on chain=going left, join=by {<-}]
            (risiko) {Risiko og gamble};
      \end{scope}
      \begin{scope}[start branch=hoejre,]
      \node (finans) [punktchain, on chain=going right] {Det finansielle system};
    \end{scope}
  \node[punktchain, join,] (disk) {Det imperfekte finansielle marked};
  \node[punktchain, join,] (makro) {Investeringsmæssige konsekvenser};
  \node[punktchain, join] (konk) {Konklusion};
  % Now that we have finished the main figure let us add some "after-drawings"
  %% First, let us connect (finans) with (disk). We want it to have
  %% square corners.
  \draw[|-,-|,->, thick,] (finans.south) |-+(0,-1em)-| (disk.north);
  \draw[|-,-|,->, thick,] (moscow.west) |-+(0,-1em)-| (krav.west);
  % Now, let us add some braches. 
  %% No. 1
  \draw[tuborg] let
    \p1=(risiko.west), \p2=(finans.east) in
    ($(\x1,\y1+2.5em)$) -- ($(\x2,\y2+2.5em)$) node[above, midway]  {Teori};
  %% No. 2
  \draw[tuborg, decoration={brace}] let \p1=(disk.north), \p2=(makro.south) in
    ($(2, \y1)$) -- ($(2, \y2)$) node[tubnode] {Analyse};
  %% No. 3
  \draw[tuborg, decoration={brace}] let \p1=(projektb.north), \p2=(litteratur.south) in
    ($(2.5, \y1)$) -- ($(2.5, \y2)$) node[tubnode] {Projektformulering};
    
  \end{tikzpicture}
  
 \textbf{Forprojekt}\\
 
 Efter udvælgelsen af projektet Synkerefleksmonitor er der blevet lavet et forprojekt. På bag grund af opgavebeskrivelsen blev der lavet et udkast til kravspecifikationen. Her blev der brugt MoSCow metoden. Metoden resultereret i hvilke krav vi skulle have løst i dette projekt, men også dem som vi måske ville kunne nå afhængig af tid og kompetencer og til slut dem vi ikke ville kunne lave, men kunne bruges til en fremtid og perspektivering af en videreudvikling af systemet. Revision historikken af MoSCoW analysen kan ses i BilagXX. Fra første udkast til den endelige version med ændringer undervejs. Den endelig Moscow analyse vil være en del af den endelige kravspecifikation. Efter MoSCow analysen var det nu muligt at lave en use case og et udkast til designsammensætningen af software og hardware, da vi kendte behovet igennem opgavebeskrivelsen. Læs første udkast af MosCow, use case, design af software og hardware (diagram) i BilagXX - forprojekt synkerefleksmonitor. 
 
Dernæst blev der udarbejdet et udkast til projektplan, hvor der blev overvejet hvilke eksperimenter og teknologier projektet ville anvende igennem hele processen. Dette udkast indeholder alt fra at ex. bygge bioimdedans på fumlebræt, indsamle synkefrekvenser til brug af Latex til rapportskrivning og google drive til dokument håndtering.

Skabeloner til mødeindkaldelser og møderefarat blev også udarbejdet og kan ses i bilagXX.
 
Vores bachelorprojekt er en videreførelse af et tideligere bachelorprojekt. Dette omhandlede at undersøge de forskellige teknologier til at måle synkefrekvensen. Så dette projekt er blevet gennemgået for at finde inspiration, erfaringer og mulige referancer herfra. Der blev også søgt efter videnskabelige artikler for at danne et overblik over muligheden for at målesynkefrekvensen. Udover den udleveret videnskabelige artikel fra opgavebeskrivelsen blev der fundet yderligere videnskabelige artikler. Se denne proces i Bilag - søgning af videnskabelige artikler. Der blev samtidtig fundet litteratur i form af undervisningsbøger igennem de forskellige semestre. (SE referance listen?)
 
Samtidig med det overstående arbejde, blev der udarbejdet en samarbejdsaftale i samarbejde med vejleder. Aftalen omhandler forventet arbejdsted og tid, roller, gruppens forventninger m.m. Den blev underskrevet og kan ses i Bilag XX - Samarbejdsaftale. Se første udkast af samarbejdsaftalen i Bilag XX - forprojekt Synkerefleksmonitor.

Der blev udarbejdet en overordnet tidsplan for hele bachelorprojektet. Opbygning er efter ASE-modellen og der er derfor muligt at se hvilke områder man skal arbejde i, men samtidig en mulighed for at ændre tidsplanen bachelorprojektet skrider frem. Der blev valgt Teamgantt som er en online portal med et gantt skema som alle gruppens medlemmer har afgang til. Læs nærmere om brugen af TeamGantt i Bilag - Værktøjer. Revision af tidsplan kan ses i Bilag - Tidsplan. 
 
Endelig kunne vi lave en konklusion på arbejdet med forprojektet. Hvor vi konkluderet blandt andet at arbejdet med MoSCow analysen har givet en større forståelse og overblik for hvilke krav der er stillet til systemet. Forprojekt gav gruppen et billede og overblik over hvordan forløbet ville nogenlunde se ud og hvilke opgaver og krav gruppen havde stillet sig selv. 



 


\chapter{Projektledelse}
I dette bachelor projekt var der valgt ikke at have en projektleder. Der var i stedet valgt en ligefordelt og kollektiv ledelse. Det har gjort at hele gruppen har skulle have overblik i hele projektet. Samtidig har det været med til at alle i gruppen kender målet og retningen og der er tydelige kommunikation og opfølgninger med faste møder og scrum møder hver dag hvor alle deltager i gruppen. Hvilket har resulteret i at eventuelle forhindringer bliver adresseret og løst undervejs. Den fælles ledelsesstil har gjort at alle deltagerne er engageret og hele tiden kender målet og er klar til at ændre sådan, at målet opnåes på den bedste mulige måde. Da der er mange praktiske opgaver i bachelorprojekt, er der lavet rollefordeling og ansvarsområder igennem hele projektet. De specifikke roller og ansvarsområder kan ses i BilagXX - Samarbejdsaftale.



\chapter{Arbejdsfordeling}
Arbejdesbyrden var ligeligt fordelt i mellem gruppemedlemmerne. Efter interesse og kompetance var det muligt at byde ind på de områder man fandt interssant. Hvor den generelle rapportskrivning af afsnit blev tildelt efter hvem der havde tid og hvad der var bedst for bachelorprojektet. Administrationen  af opgaver blev oprettet og styret fra online portalen Pivotal Tracker. Her blev der i fællesskab oprettet opgaver efter pointsystem om hvor vigtig opgaven var. nu var det så muligt at tage opgaver og udføre de, indenfor for et sprint på en uge. Efter en uge ville man kunne få et overblik over afsluttet opgave i det pågældende sprint. En nærmere beskrivelse og brugen af Pivotal Tracker findes i BilagXX - Værktøjer. 





\chapter{Planlægning}
I den overordnet planlægning blev der bruget online portalen Teamgantt. Her blev der oprettet en kalender over hele forløbet delt op i forskellige områder fra ASE-modellen. Denne kalender havde hele gruppen adgang til og mulighed for at rette løbende gennem hele projektet. Gruppens brugen af Teamgantt og indstillinger kan læses nærmere i BilagXX - værktøjer. For at opreteholde en historik over tidsplanen blev der oprettet en versions historik af tidsplanen, denne kan ses i BilagXX - tidsplan. 

Det daglige arbejde blev nedskrevet i logbogen ved dagens slutning. Logbogen blev også brugt til at notere beslutninger, ændringer og større arbejde ifm. projektet. For at læse den komplette logbog se BilagXX - Logbog.




\chapter{Projektadministration}
Rapporten og bilag er skrevet i Overleaf i tekstsproget Latex. Hvor alle afsnit er delt op i selvstændige .tex filer. Disse afsnit er delt op i en rapport og bilags mappe. Figurer brugt i projektet har også en selvstændig mappe. Møder, referater og logbog er også skrevet i Overleaf og bliver oprettet som selvstændige PDF filer. Tidsplan og opgaver er oprettet og ligger i selvstændige online værktøjerne: Teamgantt og Pivotal Tracker.    


\chapter{Møder}
\section{Interne møder i gruppen}
For at opretholde overblikket og en fast struktur i projektet var der faste møder i gruppen. Hver morgen startet med scrum møde, hvor hvert gruppe medlem fremførte igangværende opgaver og problematikker. Fredagsmødet blev holdt for at kigge tilbage på ugen og det afsluttede sprint. 

\section{Vejledermøder}
Der blev afholdt vejledermøde en gang om ugen. Disse blev brugt til at afstemme fremskridt og opståede spørgsmål i løbet af projektet.

\section{Eksternemøder}
For at udvikle et relevant slutprodukt, blev slutbrugeren løbende inddraget med møder på Hammel Neuro Center med mulighed for feedback og efterfølgende justring af produktets udvikling og mål af gruppen på bag grund af disse. 


\chapter{Konflikthåndtering}
Ved konflikter og samarbejdsproblemer løste parterne problemerne indbyrdes. Ved evt. større konflikter og manglende bidragelse til projektarbejdet inddrages projektvejleder til at hjælpe med den pågældende konflikt. 

\bibliography{library}




































\section*{Tekniske udviklingsværktøjer}
\begin{itemize}
\item ASE-modellen
\item V-modellen
\item SysML
\item UML
\item applikationsmodel
\item domænemodel
\item use case
\item Visio
\item Matlab
\item Overleaf
\item Pivotal tracker
\item Mendeley
\end{itemize}






\section*{Procesværktøjer}
\begin{itemize}
\item TeamGantt
\item Pivotal Tracker
\item samarbejdsaftaler
\item Arbejdsfordeling
\item Planlægning
\item Møder
\item Projektledelse
\item Projektsadministation
\item Google drive
\item Facebook
\item Onenote
\item GradePro
\item 

\end{itemize}







\end{document}



