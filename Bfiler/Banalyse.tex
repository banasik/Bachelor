\input{setup}

\begin{document}

\begin{titlingpage}
\begin{center}

~ \\[3cm]

%\includegraphics[width=0.6\textwidth]{figurer/ASE}~\\[1cm]

\textsc{\LARGE Bilag 4}\\[1.5cm]

%\textsc{\Large Sundhedsteknologi}\\
%\textsc{\Large 3. semesterprojekt}\\[0.5cm]

\noindent\makebox[\linewidth]{\rule{\textwidth}{0.4pt}}\\
[0.5cm]{\Huge Analyse}
\noindent\makebox[\linewidth]{\rule{\textwidth}{0.4pt}}
\end{center}
\vfill
\begin{center}
{\large 19. december 2017}
\end{center}
\end{titlingpage}

\newpage
\tableofcontents*
\newpage


\chapter{Indledning}


I afsnittet analyse vil der blive beskrevet de overvejelser om mulige løsninger i projektet og hvilke der har valgt at gå videre og begrundelse herom. Der er beskrevet og diskuteret valget af hardware- og software komponenter som er kritiske for systemet. Det udleveret diagram som kan ses på figur \ref{fig:BIdiagram} er en simple BI måler som består af en instrumenterings forstærker, en strømgenerator og udgang til elektroder. Det fungere ved at kredsløbet blev tilført et signal fra en funktiongenerator, som resulteret i en konstant genereret strøm over elektroderne. Den kendte strøm og ved at måle spændingsfaldet over elektroderne, kunne man ved brug af ohms lov $(\frac{U}{I}=R)$ beregne og vise impedansen for et synk.

\begin{figure}[H]
\centering
{\includegraphics[width=\linewidth]
{Figure/BIdiagram}}
\caption{BI diagram\cite{Aroom2009}}
\label{fig:BIdiagram}
\end{figure}


\chapter{Det oprindelige kredsløb}
\section{Hardware}

Det oprindeligekredsløb var bygget som vist på figur \ref{fig:oprindeligebd} i simuleringsværtøjet Multisim og efterfølgende på et fumlebræt. 

\begin{figure}[H]
\centering
{\includegraphics[width=6cm]
{Figure/oprindeligebd}}
\caption{Blokdiagram over det oprindelige kredsløb\cite{Aroom2009}}
\label{fig:oprindeligebd}
\end{figure}

\subsection{Strømforsyning}
I artiklen \cite{Aroom2009} blev der brugt en $\pm$12V strømforsyning tilsluttet til netforsyningen. Vi valgte at undgå netforsyningen, ved at sætte otte AA batterier i serie, både for +12V og -12V.

\begin{figure}[H]
\centering
{\includegraphics[width=10cm]
{Figure/12vbatteri}}
\caption{$\pm$12v batteriforsyning}
\label{fig:12vbatteri}
\end{figure}

\subsection{Funktionsgenerator}
Analog Discovery blev brugt som funktionsgenerator, da denne er nem og hurtig til at generere signaler. Den ønskede frekvens på 50 kHz blev brugt, da det er en brugt frekvens når der skal måles et synk\cite{Kusuhara2004}. Amplituden blev sat til 2V. 

\begin{figure}[H]
\centering
{\includegraphics[width=6cm]
{Figure/analogdis}}
\caption{Analog discovery som funktionsgenerator}
\label{fig:analogdis}
\end{figure}


\subsection{Forstærkning}
Signalet fra Analog Discovery gik ind til instrumentationsforstærkeren INA128 se figur \ref{fig:ina128}. Den tilhørende gain modstand på 51 kohm blev også brugt, da det giver en fordobling i forstærkning. På udgangen af INA128 var der nu 4V. Udover at forstærke signalet vil det også fået reduceret common-mode støj, så signalet er mindre støjfyldt.

\begin{figure}[H]
\centering
{\includegraphics[width=10cm]
{Figure/ina128}}
\caption{Diagram over INA128}
\label{fig:ina128}
\end{figure}


\begin{figure}[H]
\centering
{\includegraphics[width=10cm]
{Figure/ina128gain}}
\caption{Oversigt over hvilke frekvenser INA128 kan arbejde indenfor ved bestemte gains}
\label{fig:ina128gain}
\end{figure}

Ved en gain på 2 kan der aflæses i figur \ref{fig:ina128gain}, at båndbredden var over 100 kHz, hvilket er indenfor den ønskede frekvens på 50 kHz. 

\subsection{Strømgenerator}

\begin{figure}[H]
\centering
{\includegraphics[width=10cm]
{Figure/howland1}}
\caption{Diagram over VCCS. Den faste spænding på 4V til VCCS giver en fast strøm på 100uA}
\label{fig:howland1}
\end{figure}

Spændningen kommer ind ved modstand R1 og strømmen ud på ben 3 på LF412N. Kombinationen af bestemt ohmsk modstand størrelse og lav \% tolerance modstand giver den faste strøm. Her er der brugt 1\% modstande. Ved at ændre modstand R5, kan en ønsket strøm beregnes vha. af formlen\cite{Aroom2009}: $$I_{tissue}=2*\frac{V_{in}}{R_{5}}$$


\subsection{Elektroder}

\begin{figure}[H]
\centering
{\includegraphics[width=12cm]
{Figure/elektroder}}
\caption{Ved målingerne er der blevet brugt EKG elektroder (venstre) og EMG elektroder (højre).}
\label{fig:elektroder}
\end{figure}

De forskellige elektroder kan ses i figur \ref{fig:elektroder}. EKG elektroderne er nemme at påsætte og indeholder meget gel som giver optimal kontakt, men fysisk fylder de meget. EMG elektroderne har mindre gel, men fylder næsten ingen ting. 






\subsection{A/D-konverter}

Analog er tilsluttet en pc via usb og det analogt signal blev samplet ved at måle over elektroderne. Der blev også monteret et multimeter i serie for at aflæse den konstante strøm.


\begin{figure}[H]
\centering
{\includegraphics[width=6cm]
{Figure/adkonverter}}
\caption{Scope channel 1 positiv blev brugt til at måle spændingen over elektroderne}
\label{fig:adkonverter}
\end{figure}

\begin{figure}[H]
\centering
{\includegraphics[width=6cm]
{Figure/elektroderdia}}
\caption{Der blev målt over elektroderne mellem ben 3 på LF412N og ground.}
\label{fig:elektroderdia}
\end{figure}

\section{Software}
\subsection{Waveforms}

I programmet Waveforms kan frekvens og amplitude indstilles og resultatet kan ses i oscilloskopet.

\begin{figure}[H]
\centering
{\includegraphics[width=14cm]
{Figure/waveforms}}
\caption{Brugerinterfacet i Waveforms, hvor funktionsgeneratoren og A/D-konverter indstilles. }
\label{fig:waveforms}
\end{figure}











\section{Testopstillinger}

Kredsløbet blev bygget i simuleringsprogrammet multisim og på et fumlebræt. Begge med samme udgangspunkt som i figur \ref{fig:testopstilling1}. I de kommende testopstillinger vil der blive bekræftet systemts virkning op i mod dokumentationen fra den oprindelige artikel og andre metoder fra andre artikler.
\subsection{Testopstilling 1}




I testopstilling 1 blev kredsløbet bygget efter figuer \ref{fig:testopstilling1} først i multisim og bagefter på et fumlebræt. Testen og resultaterne blev holdt op i mod dokumentationen fra den oprindelige artikel, som kan ses i figur \ref{fig:oprindeligeonload} og \ref{fig:oprindeligerms}. Ved at sammenligne resultaterne var det muligt at se om kredsløbet opførte sig korrekt og om det kunne bruges i den videre udvikling af synkerefleksmonitor. 

\begin{figure}[H]
\centering
{\includegraphics[width=\linewidth]
{Figure/testopstilling1}}
\caption{Diagram over testopstilling 1 på baggrund af det oprindelige kredsløbsdesign.}
\label{fig:testopstilling1}
\end{figure}

\begin{figure}[H]
\centering
{\includegraphics[width=10cm]
{Figure/oprindeligenoload}}
\caption{No-Load strøm respons af VCCS fra den oprindelige artikel\cite{Aroom2009}.}
\label{fig:oprindeligeonload}
\end{figure}

\begin{figure}[H]
\centering
{\includegraphics[width=10cm]
{Figure/oprindeligerms}}
\caption{Målte spændinger over elektroderne med en vævsmodel påsat  fra den oprindelige artikel\citep{Aroom2009}.}
\label{fig:oprindeligerms}
\end{figure}

\subsubsection{Simulering}

Opstiling af simuleringen af testopstilling 1 kan ses i figur \ref{fig:testopstilling1multisim}. 

\begin{figure}[H]
\centering
{\includegraphics[width=10cm]
{Figure/testopstilling1multisim}}
\caption{Diagram over testopstilling 1 i multisim på baggrund af det oprindelige kredsløbsdesign, dog uden instrumentationsforstærker.}
\label{fig:testopstilling1multisim}
\end{figure}

\textbf{No-Load}\\
Til at bekræfte No-Load responset blev funktionsgeneratoren sat til 4V og 100Hz. På udgangen sad amperemeter for at kunne aflæse den konstante strøm.

Det kunne nu måles at den konstante strøm er på 49uA ved 100Hz, som det fremgår af figur \ref{fig:testopstilling1multisimnoload} hvilket stemmer fint overens med figur \ref{fig:oprindeligeonload} fra den oprindelige artikel. Ved at foretage flere målinger ved at varierer frekvensen kan der tegnes en graf til sammenligning.

\begin{figure}[H]
\centering
{\includegraphics[width=10cm]
{Figure/testopstilling1multisimnoload}}
\caption{Diagram over testopstilling 1 i multisim ved 4V og 100Hz, kan den konstante strøm aflæses til 49uA.}
\label{fig:testopstilling1multisimnoload}
\end{figure}

I tabel \ref{table:frekvensernoload} kan de brugte frekvenser ses og går fra 100Hz til 20kHz med et passende interval. På baggrund af disse målinger kan der laves en graf over strøm responset som i \ref{fig:oprindeligeonload}.

\begin{table}[H]
\centering
\begin{tabular}{| r | r || r | r || r | r || r | r |}
    \hline
    \textbf{Hz} & \textbf{uA} & \textbf{Hz} & \textbf{uA} & \textbf{Hz} & \textbf{uA} & \textbf{Hz} & \textbf{uA}\\ \hline
    100 & 49,19 & 2000 & 27,99 & 30000 & 27,73 & 400000 & 27,87  \\ \hline
    200 & 40,74 & 3000 & 27,84 & 40000 & 27,73 & 500000 & 27,93  \\ \hline
    300 & 35,68 & 4000 & 27,79 & 50000 & 27,73 & 600000 & 27,99  \\ \hline
    400 & 32,90 & 5000 & 27,77 & 60000 & 27,73 & 700000 & 28,07  \\ \hline
    500 & 31,30 & 6000 & 27,76 & 70000 & 27,74 & 800000 & 28,16  \\ \hline
    600 & 30,32 & 7000 & 27,75 & 80000 & 27,74 & 900000 & 28,25  \\ \hline
    700 & 29,69 & 8000 & 27,75 & 90000 & 27,74 & 1000000 & 28,32  \\ \hline
    800 & 29,26 & 9000 & 27,74 & 100000 & 27,74 &  &   \\ \hline
    900 & 28,95 & 10000 & 27,74 & 200000 & 27,78 &  &   \\ \hline
    1000 & 28,73 & 20000 & 27,73 & 300000 & 27,82 &  &  \\ \hline
\end{tabular}
    \caption{Målt strøm over elektroderne ved bestemte frekvenser.}
    \label{table:frekvensernoload}
\end{table} 


\begin{figure}[H]
\centering
\includegraphics[width=10cm]{Figure/testopstilling1multisimnoloadgraf}
\captionof{figure}{Resultatet af den målte strøm ved varieret frekvenser, som kan sammenlignes med figur \ref{fig:oprindeligeonload}. X aksen er i logaritmisk skala.}
\label{fig:stromfrekvensoprindelig}
\end{figure}


\textbf{Målte spænding}

Den målte spænding måles over elektroderne og ved at tilføje en vævsmodel som i figur \ref{fig:testopstilling1multisimvaevs}, vil spændingen ændre sig ved forskellige frekvenser. Vævsmodellen bruges til at vertificere nøjeagtighed og repeterbarhed af kredsløbet\cite{Aroom2009}.

\begin{figure}[H]
\centering
{\includegraphics[width=4cm]
{Figure/testopstilling1multisimvaevs}}
\caption{Vævsmodel med to modstande og en kondensator, som viser en elektrisk model over et væv.}
\label{fig:testopstilling1multisimvaevs}
\end{figure}




\begin{figure}[H]
\centering
{\includegraphics[width=10cm]
{Figure/testopstilling1multisimRMS}}
\caption{Diagram over testopstilling 1 i multisim ved 4V og 100Hz, hvor spændingen kan aflæses over elektroderne.}
\label{fig:testopstilling1multisimnoload}
\end{figure}

\begin{table}[H]
\centering
\begin{tabular}{| r | r || r | r || r | r |}
    \hline
    \textbf{Hz} & \textbf{VRMS} & \textbf{Hz} & \textbf{VRMS} & \textbf{Hz} & \textbf{VRMS}\\ \hline
    100 & 0,626 & 2000 & 0,047 & 30000 & 0,041 \\ \hline
    200 & 0,311 & 3000 & 0,044 & 40000 & 0,041   \\ \hline
    300 & 0,195 & 4000 & 0,043 & 50000 & 0,041   \\ \hline
    400 & 0,141 & 5000 & 0,042 & 60000 & 0,041   \\ \hline
    500 & 0,112 & 6000 & 0,042 & 70000 & 0,041  \\ \hline
    600 & 0,094 & 7000 & 0,042 & 80000 & 0,041   \\ \hline
    700 & 0,083 & 8000 & 0,041 & 90000 & 0,041  \\ \hline
    800 & 0,074 & 9000 & 0,041 & 100000 & 0,041   \\ \hline
    900 & 0,068 & 10000 & 0,041 &  &     \\ \hline
    1000 & 0,064 & 20000 & 0,041 & &   \\ \hline
\end{tabular}
    \caption{Målt VRMS ved bestemte frekvenser.}
    \label{table:frekvenservrms}
\end{table} 


\begin{figure}[H]
\centering
{\includegraphics[width=10cm]
{Figure/testopstilling1multisimVRMSgraf}}
\caption{Grafen viser de plottet frekvenser, som kan sammenlignes med figur \ref{fig:oprindeligerms} fra den oprindelige artikel.}
\label{fig:testopstilling1multisimVRMSgraf}
\end{figure}








\subsubsection{Fumlebræt}




\begin{figure}[H]
\centering
{\includegraphics[width=8cm]
{Figure/oprindeligekredslob}}
\caption{Billede af testopstilling 1 på fumlebræt på baggrund af det oprindelige kredsløbsdesign.}
\label{fig:oprindeligekredslob}
\end{figure}

\textbf{No-Load}\\

\begin{table}[H]
\centering
\begin{tabular}{| r | r || r | r || r | r || r | r |}
    \hline
    \textbf{Hz} & \textbf{uA} & \textbf{Hz} & \textbf{uA} & \textbf{Hz} & \textbf{uA} & \textbf{Hz} & \textbf{uA}\\ \hline
    100 & 49,19 & 2000 & 27,99 & 30000 & 27,73 & 400000 & 27,87  \\ \hline
    200 & 40,74 & 3000 & 27,84 & 40000 & 27,73 & 500000 & 27,93  \\ \hline
    300 & 35,68 & 4000 & 27,79 & 50000 & 27,73 & 600000 & 27,99  \\ \hline
    400 & 32,90 & 5000 & 27,77 & 60000 & 27,73 & 700000 & 28,07  \\ \hline
    500 & 31,30 & 6000 & 27,76 & 70000 & 27,74 & 800000 & 28,16  \\ \hline
    600 & 30,32 & 7000 & 27,75 & 80000 & 27,74 & 900000 & 28,25  \\ \hline
    700 & 29,69 & 8000 & 27,75 & 90000 & 27,74 & 1000000 & 28,32  \\ \hline
    800 & 29,26 & 9000 & 27,74 & 100000 & 27,74 &  &   \\ \hline
    900 & 28,95 & 10000 & 27,74 & 200000 & 27,78 &  &   \\ \hline
    1000 & 28,73 & 20000 & 27,73 & 300000 & 27,82 &  &  \\ \hline
\end{tabular}
    \caption{Målt strøm over elektroderne ved bestemte frekvenser.}
    \label{table:frekvensernoload}
\end{table} 



\textbf{Målte spænding}\\






\section{Konklusion}

\chapter{Det modificeret kredsløb}
Erfaringerne fra det oprindelige kredsløb og metoder fra andre artikler blev testet for til slut at kunne vælge det endelige videre system i projektet. Den overordnet ændring er at hardware blev delt op i to dele, en strømgenerator og spændingsmåler. Denne løsning er blevet brugt i flere artikler, \cite{Nahrstaedt2012a}, \cite{Chester}.





\section{Hardware del 1 - Strømgenerator}

I denne hardware del 1 blev der genereret strøm til to elektroder. I figur \ref{fig:analyse1} kan de enkelte komponenter ses.


\begin{figure}[H]
\centering
{\includegraphics[width=\linewidth]
{Figure/analyse1}}
\caption{Forløbet over generationen af den faste strøm.}
\label{analyse1}
\end{figure}

\subsection{Strømforsyning}
Strømforsyningen er blevet øget fra $\pm$12 til $\pm$18 da dette giver en højere excitationsspænding som bidrager til en øget strøm som kan genereres.
\begin{figure}[H]
\centering
{\includegraphics[width=8cm]
{Figure/18v}}
\caption{Ved brug af fire 9V batterier kan excitationsspænding komme op på $\pm$18V.}
\label{fig:18v}
\end{figure}

\subsection{Funktionsgenerator}
Signalet fra funktionsgeneratoren blev øget til 4V og bibeholdt 50kHz.
\subsection{Forstærkning}
Forstærkningen blev nu øget fra 4V til 8V strømgeneratoren.

\subsection{Strømgenerator}
Der vælges at øge strømmen til 500uA ved at ændre R5 til 2k, da artiklerne \cite{Chester}, \cite{Chester2014} og \cite{Kusuhara2004} bruger denne strøm til at detektere BI over svælget. 





\subsection{Elektroder}

Der blev testet med begge elektroder fra figur \ref{fig:elektroder} og med forskellige placeringer. Strømmen og den målte spændingen er nu blevet ført over sine egne ledninger. BI er bedst at måle med fire elektroder, for at undgå utilsigtet inklusion af elektrode impedans ved kun brug af to elektroder \cite[s. 420-421]{Holder2004}.


\begin{figure}[H]
\centering
{\includegraphics[width=8cm]
{Figure/elektrodeplacering1}}
\caption{Der er prøvet med forskellige elektrode placeringer. Hvor strøm elektroder er yderst og spændingen måles inderst.}
\label{fig:elektroder}
\end{figure}



\begin{figure}[H]
\centering
{\includegraphics[width=8cm]
{Figure/BIbasic}}
\caption{Diagram for hvordan man måler BI, med en fast strøm hvor spændingen kan måles over.}
\label{fig:analyse2}
\end{figure}


\section{Hardware del 2 - Spændingsmåler}
\begin{figure}[H]
\centering
{\includegraphics[width=\linewidth]
{Figure/analyse2}}
\caption{Bioimpedans ud}
\label{fig:analyse2}
\end{figure}


\subsection{Strømforsyning}

Da strømforsyningen var øget til $\pm$18V for del 1, blev del 2 forsynet med den samme excitationsspænding. 

\subsection{Elektroder}

Der blev testet med begge elektroder fra figur \ref{fig:elektroder} og med forskellige placeringer.

\subsection{Forstærkning}
Da det var en lille spænding der måltes blev den forstærket op samtidig med at støj blev reduceret. Der blev stadig holdt øje om båndbredden var indenfor hvad INA128 kunne leverer ved forskellige gains. Gain blev sat til 10, hvilket der ok som det kan aflæses i figur \ref{fig:ina128gain}.


\subsection{Antialiaseringsfilter}
Antialiaseringsfilteret består af et lavpas filter med en knækfrekvens på 50kHz, da synket er blevet amplitudemoduleret ved denne frekvens. Lavpasfilteret blev et 2.ordenfilter da det ville dæmpe med -40dB pr. dekade, således at ved 500kHz er det dæmpet -40dB.



\subsection{A/D konverter}
Når signalet er blevet forstærket til det ønskede spændingsniveau som er brugbart for A/D konverteren, er det muligt at sample signalet. Ved at vælge en høj samplingfrekvens på 1MHZ, fik vi samplet det dobbelte af det halve af nyquist frekvens.

\section{Software}
\subsection{Matlab}

\section{Testopstillinger}
\subsection{Testopstilling 1}
\subsection{Testopstilling 2 - LM318N}
\subsection{Testopstilling 3 - OPAMP}
\begin{figure}[H]
\centering
{\includegraphics[width=\linewidth]
{Figure/modificeretkredslob}}
\caption{}
\label{modificeretkredslob}
\end{figure}

\section{Konklusion}

\chapter{EMG}






\begin{figure}[H]
\centering
{\includegraphics[width=\linewidth]
{Figure/analyse1}}
\caption{Bioimpedans ud}
\label{fig:analyse1}
\end{figure}

Instrumenterings forstærker 1\\
I det oprindelig design af BI konstateret vi at det var lavet lavet til at måle BI'er på skalpen og ikke over svælget. Derfor valgte vi at instrumenterings forstærkeren fik et større signal ind fra Analog Discovery på 2V og 50kHz. I det hele taget undrede vi over artiklens valg af brug af instrumenterings forstærker i starten af kredsløbet, da den ikke er et must for at realisere kredsløbet. Men dens eneste formål var at nedbringe common-mode støj fra funktions generatoren, så vi valgte at beholde denne da vi også vil undgå så meget støj som muligt videre i kredsløbet. Gain var oprindeligt sat til 51 Kohm hvilket giver det dobbelte af hvad instrumenterings forstærkeren tilføre. I diagrammet på figur \ref{BIdiagram} kan det ses at instrumenterings forstærkeren bliver forsynet med +12/-12 V, men der er her valgt at -12 V skal direkte til ground, hvilket har resulteret i at instrumenterings forstærkeren ikke fungerer korrekt, så der er den i stedet forsynet med -12 V og ikke ground.  





Strømgenerator\\
Det forstærket signal som kommer fra udgangen på instrumenterings forstærkeren løber over til strømgeneratoren. Denne strømgenerator er en Howland bridge. Sammensætningen af modstandene er vigtige og deres tolerance skal være lav for at få en korrekt og konstant strøm. For at justerer strømmen kan R5 udskiftes i kredsløbet. For at få en konstant strøm omkring ca. 500 uA, er modstanden ændret fra 51 Kohm til 2 Kohm.  

\textbf{Det oprindelige kredsløb}\\
Først bygges det oprindelige kredsløb som det er opgivet og der bliver foretaget en no load test, for at se om det stemmer overens med figuren fra artiklen.

\begin{figure}[H]
\centering
\begin{minipage}{.5\textwidth}
  \centering
  \includegraphics[width=.9\linewidth]{Figure/VCCSwavegen1}
  \captionof{figure}{A figure}
  \label{fig:test1}
\end{minipage}%
\begin{minipage}{.5\textwidth}
  \centering
  \includegraphics[width=.9\linewidth]{Figure/oprindeligekredslob}
  \captionof{figure}{Another figure}
  \label{fig:test2}
\end{minipage}
\end{figure}








\textbf{Det modificeret kredsløb}\\

\begin{table}[H]
\begin{minipage}[b]{0.30\linewidth}
\centering
\begin{tabular}{ r |  r }
    \hline
    Hz & uA \\ \hline
    100 & 1268 \\ \hline
    200 & 1051 \\ \hline
    300 & 920 \\ \hline
    400 & 845 \\ \hline
    500 & 802 \\ \hline
    600 & 775 \\ \hline
    700 & 756 \\ \hline
    800 & 744 \\ \hline
    900 & 735 \\ \hline
    1000 & 728 \\ \hline
    2000 & 703 \\ \hline
    3000 & 696 \\ \hline
    4000 & 692 \\ \hline
    5000 & 688 \\ \hline
    6000 & 685 \\ \hline
    7000 & 683 \\ \hline
    8000 & 680 \\ \hline
    9000 & 678 \\ \hline
    10000 & 676 \\ \hline
    20000 & 675 \\ \hline
    30000 & 634 \\ \hline
    40000 & 596 \\ \hline
    50000 & 542 \\ \hline
    60000 & 475 \\ \hline
    70000 & 405 \\ \hline
    80000 & 332 \\ \hline
    90000 & 268 \\ \hline
    100000 & 210 \\ \hline
    110000 & 161 \\ \hline
    120000 & 120 \\ \hline
    130000 & 87 \\ \hline
    140000 & 60 \\ \hline
    150000 & 40 \\ \hline
    160000 & 25 \\ \hline
    170000 & 16 \\ \hline
    180000 & 10 \\ \hline
    190000 & 6 \\ \hline
    200000 & 4 \\ \hline
    210000 & 2 \\ \hline
    220000 & 1 \\ \hline
    230000 & 1 \\ \hline
    240000 & 0 \\ \hline
        
\end{tabular}
    \caption{Student Database}
    \label{table:student}
\end{minipage}\hfill
\begin{minipage}[b]{0.7\linewidth}
\centering
\includegraphics[width=10cm]{Figure/stromfrekvensoprindelig2k}
\captionof{figure}{2-D scatterplot of the Student Database}
\label{fig:image}
\end{minipage}
\end{table}














Overvejelser om mulige løsninger
løsninger I har valgt, begrundelsen herfor
grundlæggende valg af hardware- og softwaremæssige komponenter, som er kritiske for realisering af systemet

For at træffe et valg kan der analyseres og diskuteres forskellige løsninger mht. til ydeev-ne, pris, leveringstid og forhåndskendskab. Disse kan med fordel opstilles i tabelform.

Anti-alisering
Elektroder
Konstant strøm
Lavpas filtering
Ensretter
Sampling af signal \cite{martin}





\chapter{Konklusion}
\bibliography{library}

\end{document}
