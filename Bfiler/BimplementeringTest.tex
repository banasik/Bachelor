\input{setup}
\begin{document}
\begin{titlingpage}
\begin{center}

~ \\[3cm]

%\includegraphics[width=0.6\textwidth]{figurer/ASE}~\\[1cm]

\textsc{\LARGE Bilag 7	}\\[1.5cm]

%\textsc{\Large Sundhedsteknologi}\\
%\textsc{\Large 3. semesterprojekt}\\[0.5cm]

\noindent\makebox[\linewidth]{\rule{\textwidth}{0.4pt}}\\
[0.5cm]{\Huge Implementering \& Test}
\noindent\makebox[\linewidth]{\rule{\textwidth}{0.4pt}}
\end{center}
\vfill
\begin{center}
{\large 19. december 2017}
\end{center}
\end{titlingpage}

\newpage
\tableofcontents*
\newpage

\chapter{Indledning}
I dette bilag beskrives, hvordan de designede hardware og software-komponenter er implementeret og modultestet. På baggrund af disse designede enheder i designfasen foretages der nu implementering og modultest. Til implementering af hardware komponenter er der valgt at opbygge testkredsløbet på et fumlebræt. Hver af disse komponenter skal igennem en modultest for at verificere om de kan bruges til det tiltænkte formål. For softwarens vedkommende dokumenteres, hvordan kodeimplementeringen er foregået, samt resultatet af disse implementeringer. Formålet med modultesten er at forberede produktet til integrationstest.       
\chapter{Implementering}
\section{Hardware}
\subsection{Instrumentationsforstærker 1}
Denne instrumentationsforstærker er tiltænkt til at forstærke et AC signal på 2V til 4V. Til dette formål  er der anvendt instrumentationsforstærkeren INA128 med de begrundelser, som er angivet i designafsnittet. Figur \ref{figScrip} viser INA128 og en eksterne modstand, som bruges til at fastsætte den ønskede forstærkning.  


\begin{figure}[H] 
\centering
{\includegraphics[width=10cm]
{Figure/INA128IM}}
\caption{Figuren viser, hvordan INA128 er implementeret på et fumlebræt  }
\label{figScrip}
\end{figure}



\subsection{Strømgenerator}
Strømgeneratorens funktion er at levere en konstant strøm, som sendes til et måleobjektets væv. Til implementering af denne strømgenerator er der anvendt operationsforstærkeren LM318. Figur \ref{figScrip1}
viser komponenten LM318 med de tilhørende modstande. 
\begin{figure}[H] 
\centering
{\includegraphics[width=\linewidth]
{Figure/LM318IM}}
\caption{Figuren viser, hvordan LM318 er implementeret på et fumlebræt  }
\label{figScrip1}
\end{figure}


\subsection{Instrumentationsforstærker 2}

Instrumentationsforstærker 2 bliver brugt til at forstærke biosignal fra måleobjektets og undertrykkelse støj . Til implementering af denne Instrumentationsforstærker er der igen brugt INA128. Figur \ref{figINAogSpandeler} B viser komponenten INA128 med dens tilhørende eksterne modstande. Figur \ref{figINAogSpandeler} A viser en spændingsdeler kredsløb, der er benyttet til at teste instrumentationsforstærkeren. Hvorfor der anvendes en spændingsdeler til test af INA128 henvises der til \textit{"bilag 6 - Design"}. 


 

\begin{figure}[H] 
\centering
{\includegraphics[width=\linewidth]
{Figure/INA128ogSpDelerIM}}
\caption{Figuren viser, hvordan LM318 er implementeret på et fumlebræt  }
\label{figINAogSpandeler}
\end{figure}

\subsection{OP-AMP}

Biosignalet fra måleobjektet forstærkes op i to trin. Det første trin benyttes Instrumentationsforstærker 2 og det andet trin anvendes operationsforstærkeren LM318, som forstærker signalet fra Instrumentationsforstærker 2 yderligere. Figur \ref{opamp} viser implementering af LM318 og de to modstande, som fastsætter, hvor meget forstærkning man kan få ud af operationsforstærkeren. 



\begin{figure}[H] 
\centering
{\includegraphics[width=\linewidth]
{Figure/OP-AMPIM}}
\caption{Figuren viser, hvordan LM318 er implementeret på et fumlebræt  }
\label{opamp}
\end{figure}



    
\subsection{Spektrumanalyse}

Der er behov for at foretage en spektrumanalyse, med alle komponenter sat sammen, for at vurdere hvilket krav der skal være til AA filterets dæmpning.  



\begin{figure}[H] 
\centering
{\includegraphics[width=\linewidth]
{Figure/aaspectrumimplementering}}
\caption{Billede over den samlede opstilling med komponenterne: Instrumentationsforstærker 1 \& 2, strømgenerator, OP-AP og monteret elektroder. Disse komponenter repræsentere det samlet frekvensspektrum.}
\label{aaspectrumimplementering}
\end{figure}

\subsection{AA filter}

For at undgå aliasering i det optaget signal implementeres et anti-aliaseringsfilter foran Analog Discovery. AA filteret modtager et forstærketsignal fra OPAMP, filtrere det og sender det videre til Analog Discovery.    

\begin{figure}[H] 
\centering
{\includegraphics[width=6cm]
{Figure/aafilterimplementering}}
\caption{Implementering af AA filter.}
\label{aafilterimplementering}
\end{figure}





\section{Software}

I dette afsnit beskrives de vigtigste funktioner for systemet software-del. Til implementering af disse funktioner er der brugt udviklingsværktøjet Matlab pga. følgende fordele:

\begin{itemize}
\item  Matlab understøtter styring af Analog Discovery 
\item Matlab forærer databehandlingsfunktioner som ligger klar til anvendelse
\item Matlab er god til at indlæse store mængde data med få kodelinjer
\item I Matlab kan man udvikle en brugergrænseflade nemt og hurtigt 
\item Ved brug af Matlab med Analog Discovery, kan man betjene  funktionsgeneratoren, dataopsamlingsenhed og brugergrænsefladen  fra ét sted   
\end{itemize}  

Alternativet til overnævnte fordele er at man bruger forskellige “single purpose” programmer og enheder, når man skal måle et biomedicinsk signal. \\ \\
I det følgende gennemgås de kritiske funktioner for programmet.

\pagebreak

\subsection{Funktioner}
Funktionernes nærmere beskrivelser henvises der til \textit{"bilag 6 - Design".}

\subsubsection{Synkerefleksmonitor\_OpeningFcn}
\lstinputlisting[frame=single, firstline=48, lastline=59]{matlabkode/Synkerefleksmonitor.m}




\subsubsection{Btn\_Start\_Measurements}
\lstinputlisting[frame=single, firstline=107, lastline=117]{matlabkode/Synkerefleksmonitor.m}


\subsubsection{Btn\_Save\_Measuerments}
\lstinputlisting[frame=single, firstline=98, lastline=103]{matlabkode/Synkerefleksmonitor.m}


\subsubsection{Start\_GUI} 
\lstinputlisting[frame=single]{matlabkode/Start_GUI.m}



\subsubsection{Generate\_SineWave} 

\lstinputlisting[frame=single]{matlabkode/Generate_SineWave.m}



\subsubsection{Read\_Measurements}
\lstinputlisting[frame=single]{matlabkode/Read_Measurements.m}

\subsubsection{Process\_Measurements} 
\lstinputlisting[frame=single]{matlabkode/Process_Measurements.m}




\subsubsection{Show\_Measurements} 
\lstinputlisting[frame=single]{matlabkode/Show_Measurements.m}

\subsubsection{Save\_Measurments} 
\lstinputlisting[frame=single]{matlabkode/Save_Measurements.m}



\chapter{Modultest}
Modultest af hardware-delen består af en simuleret test vha. Multisim og en praktisk test. Nogle komponenterne eksister ikke i Multisim og kræver at blive oprettet, men det er valgt at ikke bruge tid på det, da det er tidskrævende. Derfor præsenteres kun praktiske resultater for disse komponenter. I det følgende præsenteres testresultaterne for instrumentationsforstærker 1, 2 , strømgeneratoren, operationsforstærkeren og AA filteret. 

\section{Hardware}

\subsection{Instrumentationsforstærker 1}
Dette modul er testet ved at sende 2V/20kHz fra Analog Discovery (den gule kurve) igennem INA128. Det ses på figur \ref{TestINA} at de 2V bliver forstærket til 4V (turkis kurve) ved udgangen af INA128. Dette resultat stemmer overens med det beregnede resultat i designfasen.    

\begin{figure}[H] 
\centering
{\includegraphics[width=\linewidth]
{Figure/TestINA1281}}
\caption{Figuren viser resultatet af INA128, som forstærker 2V til 4V}
\label{TestINA}
\end{figure}
  \pagebreak
\subsection{Strømgenerator}

Strømgeneratoren er simuleret ved at den får 4V fra en funktionsgenerator. På baggrund af denne spænding genereres der 283uA ud af strømgeneratoren. Bemærk at figur \ref{SimTestStrom} viser strømudgangen ved no-load, hvorimod figur \ref{SimTestStromNoLoad} viser når man belaster strømgeneratoren med $ 10k\Omega$.  

\begin{figure}[H] 
\centering
{\includegraphics[width=\linewidth]
{Figure/SimuleretStromGenerator}}
\caption{Figuren viser det simuleret resultat for  strømgeneratoren ved no-load}
\label{SimTestStrom}
\end{figure}



\begin{figure}[H] 
\centering
{\includegraphics[width=\linewidth]
{Figure/SimuleretStromMedLoad}}
\caption{Figuren viser det simuleret resultat for  strømgeneratoren med load på $10k\Omega$}
\label{SimTestStromNoLoad}
\end{figure}

Det ses på figur \ref{SimTestStromNoLoad} at strømmen ikke ændrer sig, selvom man belaster kredsløbet med $ 10k\Omega	$.   



\pagebreak
Tilsvarende sendes der 4V ind i strømgeneratoren, når der skal foretages den praktiske test på fumlebræt. De 4V genereres fra Analog Discovery. Den producerede strøm måles vha. 
et amperemeter i serie på udgangen af den anvendte operationsforstærker, LM318. Det ses på figur \ref{TestStrGen} at udgangsstrømmen, som genereres af strømgeneratoren er målt til 285uA. Dette resultat afviger lidt fra det beregnede og simulerede resultat. Det vurderes at afvigelsen er så lille at den ingen betydning har for måleobjektets sikkerhed. Figur \ref{fig:Stromgeneratorload} viser strømmens stabilitet fra 1 k$\Omega$ til 10 k$\Omega$, hvilket svarer til området for load impedansen for et biologisk væv \citep{Chester2014}. Derfor er det besluttet at arbejde videre med den målte strøm. 


\begin{figure}[H] 
\centering
{\includegraphics[width=\linewidth]
{Figure/VCCStestParktisk}}
\caption{Figuren viser det praktisk strømresultat som er målt ved udgangen af strømgeneratoren.}
\label{TestStrGen}
\end{figure}

\begin{figure}[H] 
\centering
{\includegraphics[width=12cm]
{Figure/Stromgeneratorload}}
\caption{}
\label{fig:Stromgeneratorload}
\end{figure}

\subsection{Instrumentationsforstærker 2}

Til test af instrumentationsforstærker 2 er der benyttet en spændingsdeler, som får 1V fra en funktionsgeneratoren, Analog Discovery. Spændingsdelerens funktion er at reducere 1V til 10mV, som efterfølgende forstærkes af instrumentationsforstærker 2 med faktor 100 gange. Som figur \ref{TestAfINA1282} viser, kan man med instrumentationsforstærkeren INA128 opnå en forstærkning på 100. Dette resultat stemmer overens med det teoretiske resultat, som er beregnet i designfasen.  

\begin{figure}[H] 
\centering
{\includegraphics[width=12cm]
{Figure/TestAfINA1282.PNG}}
\caption{Figuren viser det praktiske udgangsspænding for INA128. Denne INA128 yder en forstærkning på 100 gange.}
\label{TestAfINA1282}
\end{figure}






\subsection{OP-AMP}

Denne operationsforstærkers opgave er at forstærke 1 V til 10 V, dvs. en forstærkning ved faktor 10. Der sendes 1V fra funktionsgeneratoren, Analog Discovery, som derefter bliver forstærket til 10 V. Figur \ref{TestAfOpAmp} viser at signalet fra funktionsgeneratoren bliver forstærket 10 gange, dvs. den målte udgangsspænding er på 10V. Hermed stemmer den målte og den teoretiske udgangsspænding af operationsforstærkeren LM318 med hinanden.  

\begin{figure}[H] 
\centering
{\includegraphics[width=14cm]
{Figure/TestOpamp.PNG}}
\caption{Figuren viser det praktisk udgangsspænding som er målt ved udgangen af operationsforstærkeren.}
\label{TestAfOpAmp}
\end{figure}


\begin{figure}[H] 
\centering
{\includegraphics[width=\linewidth]
{Figure/opampmultisim}}
\caption{Figuren viser det simuleret udgangsspænding som er målt ved udgangen af operationsforstærkeren.}
\label{opampmultisim}
\end{figure}


\subsection{Spektrumanalyse}

Ved at sammensætte alle komponenter inkl. elektroder, kan spektrumanalysen blive optaget vha. Analog Discovery. Som det kan aflæses på figur \ref{fig:aaspectrum1}, er der en dæmpning fra det samlet kredsløb på -75 dB. Dette aflæses fra den højeste peak i passbåndet til ca. 250 kHz, som er den halve samplingfrekvens. Da Analog Discovery arbejder med 14bit (skal dæmpe ned til -90 dB) kræver det en yderligere dæmpning på ca. 15 dB. Dette kunne realiseres med et 1. ordens filter, men der vælges at benytte et 2. ordens filter, for at undgå de variationer der måtte være i passbåndet i det målte signal.


\begin{figure}[H] 
\centering
{\includegraphics[width=\linewidth]
{Figure/aaspectrum1}}
\caption{Det implementeret frekvensspektrum, som viser at der er en samlet dæmpning på ca. 75dB som skal yderligere dæmpes til 90dB med et 2.ordens lavpasfilter.}
\label{fig:aaspectrum1}
\end{figure}


\subsection{AA filter}

Først er AA filteret simuleret i Multisim. Ved hjælp af Bode Plotter i Multisim, kan det aflæses ved 250 kHz at er en dæmpning på 40 dB.


\begin{figure}[H] 
\centering
{\includegraphics[width=\linewidth]
{Figure/aafilterbodemultisim}}
\caption{Resultat om filterets virkning simuleret i Multisim}
\label{fig:aafilterbodemultisim}
\end{figure}

Der er brugt værktøjet Network Analyzer i programmet Waveforms, for at teste filters virkning, se figur \ref{fig:aafiltermodultest}. Det kan konstateres at ved knækfrekvensen (25 kHz) er faseforskydningen ca. \ang{90}. Efterfølgende kan det aflæses, at ved knækfrekvensen og en dekade frem er der en dæmpning på ca. 35 dB.



\begin{figure}[H] 
\centering
{\includegraphics[width=\linewidth]
{Figure/aafiltermodultest}}
\caption{Resultat om filterets virkning fra Network Analyzer i waveforms.}
\label{fig:aafiltermodultest}
\end{figure}





\section{Software}

\subsection{Funktioner}
\subsubsection{Synkerefleksmonitor\_OpeningFcn}








\subsubsection{Btn\_Start\_Measurment}

\begin{figure}[H] 
\centering
{\includegraphics[width=10cm]
{Figure/modultestStart}}
\caption{Oprettelse af dato og tid.}
\label{fig:modultestStart}
\end{figure}


\subsubsection{Btn\_Save\_Measurment}


\subsubsection{Start\_GUI}
 
\begin{figure}[H] 
\centering
{\includegraphics[width=10cm]
{Figure/modulteststartGUI1}}
\caption{Oprettelse af dato og tid.}
\label{fig:modulteststartGUI1}
\end{figure}


\subsubsection{Start\_GUI} 
\begin{figure}[H] 
\centering
{\includegraphics[width=10cm]
{Figure/modulteststartGUI2}}
\caption{Save\_Measurements knappen er gemt fra start.}
\label{fig:modulteststartGUI2}
\end{figure}






\begin{figure}[H] 
\centering
{\includegraphics[width=\linewidth]
{Figure/modultestGUI}}
\caption{Resultat om koden i "Start\_GUI"}
\label{fig:aafiltermodultest}
\end{figure}


\subsubsection{Generate\_SineWave} 

\begin{figure}[H] 
\centering
{\includegraphics[width=3cm]
{Figure/modultestsinus2}}
\caption{Resultat om koden i "Generate\_SineWave"}
\label{fig:modultestsinus1}
\end{figure}


\begin{figure}[H] 
\centering
{\includegraphics[width=10cm]
{Figure/modultestsinus3}}
\caption{Oprettelsen til Analog Discovery}
\label{fig:modultestsinus3}
\end{figure}



\begin{figure}[H] 
\centering
{\includegraphics[width=10cm]
{Figure/modultestsinus4}}
\caption{Funktionsgenerator tilføjet.}
\label{fig:modultestsinus4}
\end{figure}

\begin{figure}[H] 
\centering
{\includegraphics[width=10cm]
{Figure/modultestsinus5}}
\caption{Variablen \textit{"handles.GS"} er oprettet i handles til brug i funktionen \textit{"Read\_Measurements"}.}
\label{fig:modultestsinus5}
\end{figure}



\begin{figure}[H] 
\centering
{\includegraphics[width=\linewidth]
{Figure/modultestsinus}}
\caption{GUI resultat om koden i "Generate\_SineWave"}
\label{fig:modultestsinus}
\end{figure}







\subsubsection{Read\_Measurements}

\begin{figure}[H] 
\centering
{\includegraphics[width=10cm]
{Figure/modultestread3}}
\caption{}
\label{fig:modultestread3}
\end{figure}


\begin{figure}[H] 
\centering
{\includegraphics[width=10cm]
{Figure/modultestread2}}
\caption{}
\label{fig:modultestread2}
\end{figure}

\begin{figure}[H] 
\centering
{\includegraphics[width=10cm]
{Figure/modultestread4}}
\caption{}
\label{fig:modultestread4}
\end{figure}

\begin{figure}[H] 
\centering
{\includegraphics[width=10cm]
{Figure/modultestread5}}
\caption{}
\label{fig:modultestread5}
\end{figure}

\begin{figure}[H] 
\centering
{\includegraphics[width=10cm]
{Figure/modultestread6}}
\caption{}
\label{fig:modultestread6}
\end{figure}


\begin{figure}[H] 
\centering
{\includegraphics[width=10cm]
{Figure/modultestread7}}
\caption{}
\label{fig:modultestread6}
\end{figure}

\begin{figure}[H] 
\centering
{\includegraphics[width=10cm]
{Figure/modultestread8}}
\caption{Session "s" er stoppet og fjernet fra workspace.}
\label{fig:modultestread6}
\end{figure}





\begin{figure}[H] 
\centering
{\includegraphics[width=\linewidth]
{Figure/modultestread}}
\caption{Resultat om koden i "Read\_Measurements"}
\label{fig:modultestread}
\end{figure}




\subsubsection{Process\_Measurements} 

\begin{figure}[H] 
\centering
{\includegraphics[width=3cm]
{Figure/modultestsinus2}}
\caption{Resultat om koden i "Generate\_SineWave"}
\label{fig:modultestsinus2}
\end{figure}

\begin{figure}[H] 
\centering
{\includegraphics[width=10cm]
{Figure/modultestprocessSignal}}
\caption{Resultat om koden i "Read\_Measurements"}
\label{fig:modultestprocessSignal}
\end{figure}

\begin{figure}[H] 
\centering
{\includegraphics[width=10cm]
{Figure/modultestprocessEnsrettet}}
\caption{Resultat om koden i "Read\_Measurements"}
\label{fig:modultestprocessEnsrettet}
\end{figure}


\begin{figure}[H] 
\centering
{\includegraphics[width=10cm]
{Figure/modultestprocessFilter}}
\caption{Resultat om koden i "Read\_Measurements"}
\label{fig:modultestprocessFilter}
\end{figure}


\begin{figure}[H] 
\centering
{\includegraphics[width=10cm]
{Figure/modultestprocesshandles}}
\caption{}
\label{fig:modultestprocesshandles}
\end{figure}







\subsubsection{Show\_Measurements} 


\begin{figure}[H] 
\centering
{\includegraphics[width=\linewidth]
{Figure/modultestshow}}
\caption{}
\label{fig:modultestshow}
\end{figure}










\subsubsection{Save\_Measurments} 


\begin{figure}[H] 
\centering
{\includegraphics[width=10cm]
{Figure/modultestsave1}}
\caption{}
\label{fig:modultestsave1}
\end{figure}

\begin{figure}[H] 
\centering
{\includegraphics[width=10cm]
{Figure/modultestsave2}}
\caption{}
\label{fig:modultestsave1}
\end{figure}

\begin{figure}[H] 
\centering
{\includegraphics[width=10cm]
{Figure/modultestsave3}}
\caption{}
\label{fig:modultestsave1}
\end{figure}

\begin{figure}[H] 
\centering
{\includegraphics[width=10cm]
{Figure/modultestsave4}}
\caption{}
\label{fig:modultestsave1}
\end{figure}


\chapter{Integrationstest}


\chapter{Accepttest}






\citep{Aroom2009}
\bibliography{library}
\end{document}
