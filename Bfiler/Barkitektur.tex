\input{setup}
\begin{document}
\begin{titlingpage}
\begin{center}

~ \\[3cm]

%\includegraphics[width=0.6\textwidth]{figurer/ASE}~\\[1cm]

\textsc{\LARGE Bilag 5}\\[1.5cm]

%\textsc{\Large Sundhedsteknologi}\\
%\textsc{\Large 3. semesterprojekt}\\[0.5cm]

\noindent\makebox[\linewidth]{\rule{\textwidth}{0.4pt}}\\
[0.5cm]{\Huge Arkitektur}
\noindent\makebox[\linewidth]{\rule{\textwidth}{0.4pt}}
\end{center}
\vfill
\begin{center}
{\large 19. december 2017}
\end{center}
\end{titlingpage}

\newpage
\tableofcontents*
\newpage

\chapter{Indledning}

I dette bilag beskrives hardware- og softwarearkitekturen for systemet, som ønskes realiseret. Formålet med arkitekturen er at definere, hvilke "roller"  de enkelte hardware-og softwareenheder skal tildeles. Når disse roller er tildelt kan man efterfølgende designe systemet i detaljer. For at illustrere arkitekturen i hardware-delen benyttes Blok definition diagram(BDD), Internal blok diagram(IBD) og en blokbeskrivelse, der indeholder uddybende beskrivelse af blokkene i BDD'et. For software-delens vedkommende benyttes også et BDD. Dette BDD bruges til at illustrere hovedblokkene som software-delen består af.       





\chapter{Hardware}
\section{Blok definition diagram}

Blok definition diagrammet på figur \ref{figbdd} viser synkerefleksmonitoren, som overordnet består af en hardware-blok og to blokke, som har relationer til HW-blokken. HW-blokken består ydeligere af to blokke, der hver repræsenterer en bioimpedans-måler (BI-måler) og en elektromyografi-måler(EMG-måler). EMG-måleren består to komponenter og BI-måleren består af en række komponenter. Funktionerne af disse komponenter kan læses i tabel \ref{tab:Sigbeskriv.}, hvor der også er beskrevet signaltyper, port- og blok-navne.  

\begin{figure}[H] 
\centering
{\includegraphics[width=\linewidth]
{Figure/BDD2}}
\caption{Figuren viser de enkelte komponenter, som hardware-delen består af. Overordnet består hardwaren af en Bioimpedansmåler og EMG-måler og en enhed, som bruges til at forsyne  målerapparaterne. DAQ'en anvendes som dataopsamlingsenhed.}
\label{figbdd}
\end{figure}

\section{Internal blok diagram}

Det interne blokdiagram på \ref{ibdfigur} viser den interne struktur og kommunikation mellem delsystemerne. Figur \ref{ibdfigur} indeholder to uafhængige blokke med navnene BI-måler og EMG-måler. De to måleapparater kommunikerer med Analog Discovery og en PC. For BI-målerens vedkommende starter kommunikationsflowet med at Analog Discovery'en sender to \textbf{2V} AC spænding til den første forstærker i BI-måler blokken. Forstærker 1 forstærker de \textbf{2V} med faktor 2. Det forstærket signal sendes videre til strømgeneratoren, VCCS, som på baggrund af det indkommende spænding producerer en konstant strøm på \textbf{0,5mA}. Strømen sendes videre til et måleobjekt via. to elektroder, kaldet Blue Sensor Electrodes.       Yderligere to elektroder påsættes på måleobjektet for at måle en spændingsforskel. Denne spændingsforskel ligger i millivolt området og kræver at blive forstærket. Denne forstærkning foregår over trin. Til formålet anvendes en instrumentationsforstærker efterfulgt af en operationsforstærker. Det forstærker signal sendes videre til et anti-aliaseringsfilter, der dæmper frekvenskomponenter over \textbf{Nyquist-frekvensen}. Tilslut sendes signalet til en dataopsamlingsenhed, kaldet DAQ NI USB-6259, der sender det opsamlede signal videre til en PC for at blive analyseret og vist. Delsystemerne forstærker 1, 2 og AA filteret forsynes med en eksitationsspænding på $ \pm  $\textbf{18V}. \\

EMG-blokken består en Myoware Muscle Sensor og to elektroder, der måler spændingsfaldet over et valgt segment på et måleobjekt. Det målte signal sendes til en PC for at blive vist. Myoware Muscle Sensorens eksitationsspænding på \textbf{5V} på  kommer fra Analog Discovery.   
					\textbf{INDSÆT EN BLOK OM ELNETET} 
\begin{figure}[H]
\centering
{\includegraphics[width=\linewidth]
{Figure/IBD2}}
\caption{Figuren viser et internt blokdiagram, der illustrer den interne relation og signalflow mellem delsystemer. Overordnet set indeholder diagrammet to hovedblokke med hver deres subkomponenter. Den ene af de store blokke repræsenter en bioimpedansmåler-apparat og den anden blok repræsenter en elektromyografi-apparat }
\label{ibdfigur}
\end{figure}

\section{Blokbeskrivelse} \label{blokbesk}
Nedenstående tabel viser den enkelte blokes funktion, signaltype og port navn, som indgår i IBD'et på figur \ref{ibdfigur}.

\begin{table} [H]
  \centering

\begin{tabular}  {|p{3cm}|p{4cm}|p{1cm}|p{1.5cm}|p{3.8cm}| }

\hline
	
	\textbf{Blok-navn} & \textbf{Funktionbeskrivelse}  & \textbf{Port} & \textbf{Signaler} & \textbf{Kommentar} \\ \cline{3-5} \hline
	
PC & Behandler input fra Analog Discovery.  &  C1 & bit(USB)  & Interfacekommunikation  \\ \cline{3-5}
	 &  & C2 & Graf & Impedans og tid \\ 
	 
	 \hline
	 
	 
Analog Discovery  & Forsyner MyoWare Muscle Sensor og Forstærker 1. Den fungerer også som  Analog-til-digital-konverter.  &  A1 & 2V & Udgangsspænding    \\ \cline{3-5}
	 &  & A2 & bit(seriel) & Indgangsspænding \\ \cline{3-5}
	 &  & A3 & bit(USB)  & Interfacekommunikation \\
 \cline{3-5}
	 &  & A4 & bit(seriel)  & Indgangsspænding 	 
	 \\
 \cline{3-5}
	 &  & A5 & 5V  & Eksitationsspænding 	 	 
	 
	 
	  \\ \hline	  
	 
		 
 Forstærker 1 & Forstærker 2V fra Analog Discovery til 4V   &  P1 & 2V & Indgangsspænding   \\ \cline{3-5}
	 &  & P2 & $   ${-18V}  & Eksitationsspænding \\ \cline{3-5}
	 &  & P3 & 4V  & Udgangsspænding   \\ 
	\cline{3-5}
	 &  & P4 & $   ${18V}  & Eksitationsspænding   
	 
	  \\ \hline	 	
	 
	 
Strømgenerator  & Genererer en konstant strøm &  J1 &$   ${18V} & Eksitationsspænding    \\ \cline{3-5}
	 &  & J2 & 4V  & Indgangsspænding \\ \cline{3-5}
	 &  & J3 & $ \pm  $\textbf{0.5mA}  & Udgangsstrøm  \\ \cline{3-5}
	
	 &  & J4 & $  ${-18V}  & Eksitationsspænding 	 
	 
	   \\ \hline 
	 
	 
 Forstærker 2 & Forstærker biosignal fra et måleobjekt   &  F1 & mV & Indgangsspænding     \\ \cline{3-5}
	 &  & F2 & $  ${18V}  & Eksitationsspænding \\ \cline{3-5}
	 &  & F3 & V  & Udgangsspænding  \\ \cline{3-5}
	 &  & F4 & $  ${-18V}  & Eksitationsspænding
	 
	   \\ \hline
	 
	 
4 x Blue Sensor Electrodes & Transporterer strøm til et måleobjekt og måler biosignal fra et måleobjekt.  &  T1 & uA & Udgangsstrøm  \\ \cline{3-5}
	 &  & T2 & uA & Udgangsstrøm \\ \cline{3-5} 
	 &  & T3 & mV & Biopotentiale \\ \cline{3-5} 
	 &  & T4 & mV & Biopotentiale \\ \cline{3-5}  \hline
	 
	 
	  OPAMP & Forstærker signalet fra forstærker 2
	 &  O1 & V & Indgangsspænding     \\ \cline{3-5}
	 &  & O2 & $  ${18V}  & Eksitationsspænding \\ \cline{3-5}
	 &  & O3 & V  & Udgangsspænding  \\ \cline{3-5}
	 &  & O4 & $  ${-18V}  & Eksitationsspænding
	 
	   \\ \hline
	 
 \end{tabular}
 
\end{table}

\pagebreak

\begin{table} [H]
  \centering

\begin{tabular}  {|p{3cm}|p{4cm}|p{1cm}|p{1.5cm}|p{3.8cm}| }

\hline

AA filter & Bruges til at undgå aliasering.  &  N1 & V & Indgangsspænding  \\ \cline{3-5}
	 &  & N2 & $  ${18V} & Eksitationsspænding \\ \cline{3-5} 
	 &  & N3 & V & Udgangsspænding \\ \cline{3-5} 
	 &  & N4 & $  ${-18V} & Eksitationsspænding \\ \cline{3-5} 
	 
	 \hline
	 

MyoWare Muscle Sensor & Behandler EMG input fra et måleobjekt 		 &  S1 & mV & Biopotentiale  \\ \cline{3-5}
	 &  & S2 & bit(seriel) & Biopotentiale \\ \cline{3-5} 
	 &  & S3 & 5V & Eksitationsspænding \\ \cline{3-5} \hline
	 
	 
2x Kendall electrodes & Transporterer EMG signal fra et måleobjekt  &  L1 & mV & Biopotentiale  \\ \cline{3-5}
	 &  & L2 & mV & Biopotentiale \\ \cline{3-5} \hline
	 

\end{tabular}
 \caption{Figuren giver overblik over blok navn, blok funktionn og signaltype af de komponenter, som indgår i det interne blokdiagram på figur \ref{ibdfigur}} \label{tab:Sigbeskriv.}
\end{table}

\textbf{Husk den anden Tabel}
\chapter{Software} \label{swafsnit}
\section{Blok definition diagram}

Dette afsnit omhandler arkitekturen af   softwaren, som anvendes til analysering og visning af bioimpedans- og EMG-målinger. Arkitekturen af softwaren er drevet af de usecases, som er beskrevet i afsnittet systembeskrivelse. På baggrund af disse usecases udformes et BDD, som indeholder en parent-blok og to child-blokke, der hver indeholder funktioner/metoder.
I dette projekt anvendes  Matlab til at realisere projektets  software-del.  Funktionaliteter som ønskes implementeret i Matlab kodes som funktioner. Disse funktioner skrives hver i en selvstændig script-fil, som kaldes fra en GUI fil, når de skal bruges. Da det ønskes i dette projekt at implementere en Matlab GUI med kontroller skrives koden til disse kontroller i funktioner. Kontrollerne kan være en knap, tekstfelt eller tekstboks. I stedet for at kode alle funktioner i én scriptfil tildeles hver funktionen sin egen scriptsession. Hver scriptsession omtales som en funktion/metode, der udfører en bestemt opgave, samt kan interagere med andre funktion. Konkret fungerer softwaren ved at et sundhedspersonale initialiserer kodeeksekveringen ved at starte programmet Synkerefleksmonitor og efterfølgende trykke knappen ’Start measurments'. Denne initiering af sundhedspersonalet medfører at der foretages to målinger simultant. Disse målinger analyseres og vises til sundhedspersonalet. Rækkefølgen hvori programmets kode eksekveres beskrives vha. et sekvens diagram. Dette diagram kan læses i designbilaget.    


\begin{figure}[H] 
\centering
{\includegraphics[width=\linewidth]
{Figure/SWIBD}}
\caption{Figuren viser block definition diagrammet for det ønsket software. Diagrammet indeholder en hovedblok, der består af to andre blokke, som hver indeholder Matlab funktioner. Disse funktioner tilsammen analyserer og viser to målinger simultant  }
\label{figScrip}
\end{figure}

\textbf{USE casene skal ændres der vi nu har en anderledes sw arkitektur}

\citep{Aroom2009}
\bibliography{library}
\end{document}