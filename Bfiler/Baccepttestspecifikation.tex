\input{setup}
\begin{document}
\begin{titlingpage}
\begin{center}

~ \\[3cm]

%\includegraphics[width=0.6\textwidth]{figurer/ASE}~\\[1cm]

\textsc{\LARGE Bilag 2}\\[1.5cm]

%\textsc{\Large Sundhedsteknologi}\\
%\textsc{\Large 3. semesterprojekt}\\[0.5cm]

\noindent\makebox[\linewidth]{\rule{\textwidth}{0.4pt}}\\
[0.5cm]{\Huge Accepttestspecifikation}
\noindent\makebox[\linewidth]{\rule{\textwidth}{0.4pt}}
\end{center}
\vfill
\begin{center}
{\large 16. december 2017}
\end{center}
\end{titlingpage}

\newpage
\tableofcontents*


\chapter{Indledning}
Accepttesten er en opfølgning af kravspecifikationen, som har til formål at sikre, at alle kravene er overholdt. Der vil blive testet både på hovedscenarier, samt undtagelser og udvidelser. Det er målsætningen, at disse test sikrer produktets kvalitet, idet produktet vil blive afprøvet før det tages i brug. Derfor er det accepttestens ansvarsfunktion, at godkende de opsatte delmål for produktet, hvad angår både funktionelle, samt ikke-funktionelle krav.
Den data der benyttes til målingerne fås fra rigtig målinger fra synkerefleksmonitoren. Brugergrænsefladen er det som sundhedspersonalet interagerer med, altså hvorfra systemet
aktiveres. Når der i feltet Godkendt er et flueben, betyder det at testen er godkendt. Hvis der er et flueben i parenteser, betyder det at den er delvis godkendt. Hvis der er et kryds betyder
det, at den ikke er godkendt.










\chapter{Accepttestspecifikation}

\section{Versionshistorik}
\begin{table}[H]

\begin{longtabu} to \linewidth{@{}l l l X[l]@{}}
    Version 	&    Dato 		&    Ansvarlig 	&    Beskrivelse\\[-1ex]
    \midrule
    0.1 		&  	27-09-2017 	&   MBA 	&   Oprettelse af Accepttestspecifikation \\
	0.2			&	27-09-2017	&	MBA \& MOH	&	Udfyldning af UC2 - UC4 og aktør kontekstdiagram tilføjet\\
    0.3			&	29-09-2017	&	MBA \& MOH	&	Udfyldning af ikke-funktionelle\\
    
\label{version_Systemark}
\end{longtabu}
 \caption {Versionshistorik}
    \label{tab:Versionshistorik}
\end{table}
	

\section{Accepttest af funktionelle krav}




\subsection{Use Case 1}
\textbf{Start BI-måling}

\begin{longtabu} to \linewidth{@{} c X[j] X[j] X[j] l@{}}
    ~ &	Test &    Forventet resultat &		Faktiske observationer &    Godkendt\\[-1ex]
    \midrule
    ~ &\textit{Hovedscenarie} & ~ & ~ &
    \\ \midrule
    1. &Tryk på knappen "Start BI-måling". &   BI-målingen begynder.  &     &		%{\Huge \checkmark}
    \\
    2. &Tryk på knappen "Stop måling" efter den ønskede måling er færdige.  &    Målingen stopper og gemmes i en fil.  &     &		%{\Huge \checkmark}
    \\
    3. &Verificer at filen er gemt.  &    Filen eksisterer i Matlab  &     &		%{\Huge \checkmark}
	\\ \midrule
	~ &\textit{Undtagelser} & ~ & ~ &
    \\ \midrule
    3a. &Systemet har ikke gemt målingen  &    Filen eksisterer ikke i Matlab &     &		%{\Huge \checkmark}
	\\ \midrule	
    
 \\ \bottomrule
 
\caption{Accepttest af Use Case 1.}\\
\label{AT_UC1}
\end{longtabu}


\subsection{Use Case 2}
\textbf{Start EMG-måling}

\begin{longtabu} to \linewidth{@{} c X[j] X[j] X[j] l@{}}
    ~ &	Test &    Forventet resultat &		Faktiske observationer &    Godkendt\\[-1ex]
    \midrule
    ~ &\textit{Hovedscenarie} & ~ & ~ &
    \\ \midrule
    1. &Tryk på knappen "Start EMG-måling" &   EMG-målingen begynder  &    &		%{\Huge \checkmark}
    \\
    2. &Tryk på knappen "Stop måling" efter den ønskede måling er færdige.  &    Måling stopper og gemmes i en fil  &     &		%{\Huge \checkmark}
     \\
    3. &Verificer at filen er gemt.  &    Filen eksisterer i Matlab  &     &		%{\Huge \checkmark}
	\\ \midrule
	~ &\textit{Undtagelser} & ~ & ~ & 
	\\ \midrule
    3a. &Systemet har ikke gemt målingen  &    Filen eksisterer ikke i Matlab &     &		%{\Huge \checkmark}
	\\ \midrule	
    
 \\ \bottomrule
 
\caption{Accepttest af Use Case 2}\\
\label{AT_UC1}
\end{longtabu}


\subsection{Use Case 3}
\textbf{Beregn BI}

\begin{longtabu} to \linewidth{@{} c X[j] X[j] X[j] l@{}}
    ~ &	Test &    Forventet resultat &		Faktiske observationer &    Godkendt\\[-1ex]
    \midrule
    ~ &\textit{Hovedscenarie} & ~ & ~ &
    \\ \midrule
    1. &Tryk på knappen "Beregn BI" &   Systemet foretager BI-beregningen og gemmes i en fil  &     &		%{\Huge \checkmark}
    \\
    2. &Verificer at filen er gemt  &    Filen eksisterer i Matlab  &     &		%{\Huge \checkmark}
	\\ \midrule
	~ &\textit{Undtagelser} & ~ & ~ & 
	\\ \midrule
    2a. &Systemet har ikke gemt BI-beregningen  &    Filen eksisterer ikke i Matlab &     &		%{\Huge \checkmark}
	\\ \midrule	
    
 \\ \bottomrule
 
\caption{Accepttest af Use Case 3}\\
\label{AT_UC1}
\end{longtabu}



\subsection{Use Case 4}
\textbf{Vis BI \& EMG}

\begin{longtabu} to \linewidth{@{} c X[j] X[j] X[j] l@{}}
    ~ &	Test &    Forventet resultat &		Faktiske observationer &    Godkendt\\[-1ex]
    \midrule
    ~ &\textit{Hovedscenarie} & ~ & ~ &
    \\ \midrule
    1. &Tryk på knappen "Vis BI \& EMG" &   Graferne vises  &     &		%{\Huge \checkmark}
    \\
    2. &Verificer at graferne vises   &    Graferne er vist  &     &		%{\Huge \checkmark}
	\\ \midrule
	~ &\textit{Undtagelser} & ~ & ~ & 
	\\ \midrule	
    
 \\ \bottomrule
 
\caption{Accepttest af Use Case 4}\\
\label{AT_UC1}
\end{longtabu}

\newpage

\section{Accepttest af ikke-funktionelle krav}
\subsection{Usability} 
\begin{longtabu} to \linewidth{@{} c X[l] X[l] X[j] X[j] l@{}}
	Krav nr. & Krav & Test & Forventet resultat & Faktiske observationer & Godkendt
	\\[-1ex] \midrule
	1. & Giv 10 minutters introduktion om Synkerefleksmontoren  & Kør Use Case 1, 2, 3 og 4 & Use Case 1, 2, 3 og 4 er kørt&  & %{\Huge \checkmark}
	\\ 
	\midrule
	
	2. & Foretage en måling uden fejl & Kør Use Case 1, 2, 3 og 4 & Måling er foretaget uden fejl &  & %{\Huge \checkmark}
	\\ 
	\midrule
	
	3. & Efter en periode på en uge, foretage en ny måling uden fejl & Kør Use Case 1, 2, 3 og 4 & Måling er foretaget uden fejl &  & %{\Huge \checkmark}
	\\ 
	\midrule
	
	4. & Giv karakter til GUI-designet på en skala fra 1-5 & Karakteren gives & Karakteren ligger i mellem 1-5 & %{\Huge \checkmark}
	\\ 
	\midrule
	
	5. & Aflæs graferne i GUI fra 2 meters afstand & Stå/sidde 2 meter fra skærmen & Graferne er læselige fra 2 meters afstand  & %{\Huge \checkmark}
	\\ 
	\midrule
    \caption{Usability test}
	\end{longtabu}
    
    
    
    
 \subsection{Performance}   
    \begin{longtabu} to \linewidth{@{} c X[l] X[l] X[j] X[j] l@{}}
    Krav nr. & Krav & Test & Forventet resultat & Faktiske observationer & Godkendt
	\\[-1ex] \midrule
    
    
	
	9. & Synkerefleks-monitoren skal kunne tændes indenfor 3 minutter & Tænd Synkerefleks-monitoren  & Synkerefleks-monitoren tændes indenfor 3 minutter &   &  %{\Huge \checkmark}
	\\ 
	\midrule
	
	
	
	10. & Synkerefleks-monitorens GUI skal kunne vises indenfor 3 minutter & Start Synkerefleks-monitoren i Matlab & Synkerefleks-monitorens GUI vises indenfor 3 minutter & &  %{\Huge \checkmark}
	\\ 
	\midrule
	
	
	
	11. & Synkerefleks-monitoren GUI skal have en respondstid på maks. 10 sek. & Tryk på GUI-knapperne og mål at respondstiden er indenfor 10 sek.& Responstiden er under 10 sek. & & %{\Huge \checkmark}
	\\ 
	\midrule
    \caption{Performance test}
    \end{longtabu}
    
\subsection{Supportability}  
    \begin{longtabu} to \linewidth{@{} c X[l] X[l] X[j] X[j] l@{}}
    Krav nr. & Krav & Test & Forventet resultat & Faktiske observationer & Godkendt
	\\[-1ex] \midrule
    
	
	12. & Udskiftning af batterier indenfor 2 minutter & Tag batterierne ud og indsæt dem igen & Batterierne bliver udskiftet på maks. 2 min &   & %{\Huge \checkmark}
	\\ 
	\midrule
	
	13. & Udskiftning af elektroder indenfor 2 min & Afmonter elektroder og påmonter dem igen & Elektroderne er udskiftet indenfor 2 min &  & %{\Huge \checkmark}
	\\ 
		
		
	\\ 
	\bottomrule
\caption{Supportability test}
\end{longtabu}


\end{document}