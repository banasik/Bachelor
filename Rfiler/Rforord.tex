\chapter{Forord}

Dette bachelorprojekt er udarbejdet af en gruppe bestående af to medlemmer, Mohammed Hussein Mohamed og Martin Banasik. Begge gruppemedlemmer læser Sundhedsteknologi og er i gang med 7. semester ved Ingeniørhøjskolen Aarhus Universitet.

Vejlederne i dette bachelorprojekt er Thomas Nielsen og Samuel Alberg Thrysøe som har fungeret som bivejleder. Begge vejledere er tilknyttet Ingeniørhøjskolen Aarhus Universitet.

Bachelorprojektets arbejde startede d. 4. september 2017  med afleveringsdato d. 19. december 2017 og forsvares d. 17. januar 2018.

Til afleveringsdatoen oploades de udviklede dokumenter på www.eksamen.au.dk som pdf og zip-fil. 

Projektgruppen vil gerne benytte lejligheden til takke følgende personer, som har været behjælpelige under dette projektforløb:

\begin{itemize}
\item	Gruppen vil gerne takke projektets vejleder Thomas Nielsen, Adjunkt ved Ingeniørhøjskolen Aarhus Universitet. Tak for dit engagement igennem  hele forløbet. Tak for de gode ideer som du bidrog med.  

\item Ligeledes vil gruppen takke projektets bivejleder Samuel Alberg Thrysøe, Associate Professor PhD ved Ingeniørhøjskolen Aarhus Universitet. Tak for lån af elektroder og EMG sensor. Tak for den overordnede vejledning.   

\item Også tak til Steven Brantlov, Specialrådgiver, ansvarlig for kvalitet og sikkerhed af medicoteknik ved RegionMidt. Tak for de gode råd. De var brugbare. Ligeledes tak for lån af bioimpedans-måler apparatet.  

\item Derudover tak til Peter Johansen, Associate Professor PhD ved Ingeniørhøjskolen Aarhus Universitet. Tak for lån af dataopsamlingsenheden. 
  
\end{itemize}

\pagebreak

\textbf{Læsevejledning}\\

Rapporten indledes med det første kapitel der indeholder en indledning, der præsenterer projektets baggrund, problemformulering og formål. Efterfølgende kommer de resterende kapitler, som er opbygget efter Ingeniørhøjskolens vejledning til udfærdigelse af projektrapporter. Se denne vejledning i \nameref{bilag19}. 

 
Projektets rækkefølge er at problemerne, der er formuleret i problemformuleringen først analyseres. Herefter bearbejdes analysens resultater og der drages en konklusion på baggrund af denne. Som et resultat af konklusions udfald, udspecificeres en række krav til den ønskede løsning. For at arbejde hen imod opfyldelsen af disse krav udarbejdes en arkitektur og design til løsningen. Denne løsning implementeres, testes og der diskuteres om de opnåede resultater under disse tests. På baggrund af de opnåede resultater udledes der en konklusion. Tilslut gives der en række fremtidige indsatsområder, der kan forbedre den endelige løsning, som er udviklet under dette projekt. 

Til rapporten medfølger bilag som dokumentation. Derudover er der vedlagt det udviklet software. 













