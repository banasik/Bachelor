\chapter{Diskussion af resultater}

I dette kapitel vil der blive diskuteret relevante dele af de opnåede resultater og deres betydning for bachelorprojektet, samt hvilke muligheder og begrænsninger der er ved anvendelse af SRM. Til slut opsummeres der med den samlet vurdering af de opnåede resultater med relation til problemformuleringen.

\section{Konstant strøm}
Ud fra de forløbende tests kom strømmen til at ændre sig. Ønsket var en konstant strøm på 283 $\mu A$. Problemet ved ikke at have en fast strøm er, at BI-signalet ikke bliver valide, da denne faste strøm bruges til at udregne impedansen af BI-signalet. Strømmen faldt til 104 $\mu A$, da elektroderne blev påmonteret måleobjektet, hvilket ikke stemmer overens med den beregnede/målte 283 $\mu A$. Da MyoWare Muscle Sensor blev tilføjet steg strømmen igen til 283 $\mu A$. Det kan skyldes at MyoWare Muscle Sensor bruger en reference elektrode, som nu har fælles stel med BI-sensoren. Dette er ikke med i designet af BI-sensoren, og derfor er der ikke taget høje for dette. En mulig tilføjelse til BI-sensoren, kunne være en reference elektrode til at stabilisere strømmen. Denne referance elektrode er også anvendt i nogle litteratur. Se denne artikel\cite{Nahrstaedt2012a}. 

\section{AA filter}
BI-signalet afhænger af et AA filter der er korrekt bygget og fungerer efter hensigten. Ellers vil den efterfølgende signalbehandling og visning af signalet være ubrugeligt pga. aliasering. Testene viser at AA filter fungerer efter hensigten ved at blive dæmpet ned til 95 dB ved 250 kHz. 


\section{Elektrode placering}
Ved brug af de store EKG elektroder var der begrænset fleksibilitet mht. placeringen af disse elektroder. Inspirationen af placeringer af eletroderne blev fundet i flere artikler.\cite{Chester}
\cite{Schultheiss2013}. Forskellige placeringer af elektroder har indflydelse på det optaget BI-signal\cite{Yamamoto2000}.  



\section{Digital signal behandling}
Da signalerne er samplet med en forholdsvis høj samplingrate på 500 kHz, giver det en stor mængde data. Med fordel kan denne store mængde data være med til, at med til at muliggør at signalbehandling af BI-signalet kan foregår  digitale domæne. Behandlingen af envelope i det digitale domæne, gjorde at det nu ikke var nødvendigt med yderligere hardware komponenter, såsom ensretter og et lavpas filter.

\section{GUI}

Figur \ref{fig:synkfraGUI} og \ref{fig:synkfraArtikel} viser et synk, optaget med SRM'en og en af de fundne litteratur. Når disse to synk sammenlignes med hinanden kan det konstateres at de samme adfærd. Både længden af synket og selve synkets bakkedal er sammenlignelige. Længden af synket kan i begge tilfælde aflæses til at vare ca. et sekund. Synkets bakkedal målt med SRM har også samme signatur. På dette grundlag er der argumenter for at det er muligt at detektere og vise et synk med SRM sammenlignet med en tilsvarende BI-sensor.

\begin{figure}[H]
\centering
\begin{minipage}{.5\textwidth}
  \centering
  \includegraphics[width=.7\linewidth]{Figure/synkfraGUI}
  \captionof{figure}{Et synk målt med SRM. X aksen viser tid (s).}
  \label{fig:synkfraGUI}
\end{minipage}%
\begin{minipage}{.5\textwidth}
  \centering
  \includegraphics[width=.7\linewidth]{Figure/synkfraArtikel}
  \captionof{figure}{Et synk målt i en artikel\cite{Schultheiss2014}.X aksen viser tid (s).}
  \label{fig:synkfraArtikel}
\end{minipage}
\end{figure}

Variationen i et synk består af en ændring af impedansen på 5$\Omega$ \cite[s. 49]{Chester2014}. I figur \ref{Fig:synkfraGUI5ohm} kan et synk målt med SRM aflæses som et ohmsk fald på ca. 5$\Omega$. Det ohmske fald lægger sig fint og af hvad litteraturen oplyser\cite[s. 49]{Chester2014}. Med en yderligere fintuning af den digitale signalbehandling af BI-signalet, ville man kunne få endnu bedre synk. Her er det parametrene såsom lavpas filter, downsample og smooth man ville kunne ændre på, for at for endnu bedre synk.

\begin{figure}[H]
\centering
{\includegraphics[width=4cm]
{Figure/synkfraGUI5ohm}}
\caption{Et synk målt med SRM, hvor synket er droppet til ca. 5$\Omega$. Dette er indenfor området for ændringen af synkets impedans.}
\label{Fig:synkfraGUI5ohm}
\end{figure} 

Algoritmen der finder peaks, altså i dette tilfælde synk, kan ses på figur \ref{Fig:synkfundet}. Disse synk illustreres med blå prikker. I algoritmen er der sat en minimum størrelse for bakkedalen og afstanden i mellem hvert synk. Da dette er en meget forsimplet algoritme, vil der ikke blive fundet synk, hvis værdierne ikke er indenfor disse områder. Dette vil give en fejltælling af antal synk.

\begin{figure}[H]
\centering
{\includegraphics[width=\linewidth]
{Figure/synkfundet}}
\caption{Algoritmen der finder antal synk i en måling og informere sundhedspersonalet om antallet i GUI.}
\label{Fig:synkfundet}
\end{figure} 



\section{EMG-sensor}

På figur \ref{Fig:emgmaling} er EMG-målingen vist. EMG-sensoren består af hardware der selv sørge for envelope og yderligere udglatning af EMG-signalet. Da EMG-signalet også er samplet ved 500 kHz giver det en mængde data man ønsker at reducere. Den nuværende signalbehandling består i at downsample EMG signal og efterfølgende smooth. Dette er måske ikke nødvendigt, men disse parametre kan også justeres til man får det ønskede resultat. 

\begin{figure}[H]
\centering
{\includegraphics[width=\linewidth]
{Figure/emgmaling}}
\caption{En EMG måling med SRM. Her kan der aflæses fire udsving på EMG signalet.}
\label{Fig:emgmaling}
\end{figure} 

Det målte EMG-signal kan sammenlignes med et EMG-signal fra MyoWare Muscle Sensor's datablad \nameref{bilag15}. EMG signalet ser ud til at have samme peak, ved hvert EMG aktivitet. Udforingen med EMG måleren var, at optil det første sekund af en måling, genererede EMG måleren en høj peak. Hvilken kan muligvis relateres til når funktionsgeneratoren starter, når der foretages en måling. Dog påvirker det ikke resten af målingen.


\begin{figure}[H]
\centering
{\includegraphics[width=\linewidth]
{Figure/emgsignal}}
\caption{Et eksampel på EMG signal fra MyoWare Muscle Sensor's datablad.}
\label{Fig:emgsignal}
\end{figure} 

Det færdige resultat af SRM kan ses på figur \ref{Fig:guiDone2}. Resultatet er simultane målinger af BI- og EMG-signaler. I BI-signalet vises antal synk, som vises ude til venstre i GUI. Det er muligt at indskrive den akutelle strøm og vælge hvor langtid en måling skal foretages. Det er samtidig muligt at foretage de ønskede handlinger vha. de tre knapper. 

\begin{figure}[H]
\centering
{\includegraphics[width=\linewidth]
{Figure/guiDone}}
\caption{Det komplette interface af SRM.}
\label{Fig:guiDone2}
\end{figure} 



Den samlet vurdering af de opnåede resultater med relation til problemformuleringen kan opstilles i punktform. Disse punkter opsummerer resultater som vi er særlig stolte af. 

\begin{center}
\begin{large}
\begin{tabular}{|l l|}
\hline
-Udviklet funktionel Bioimpedans-sensor&\checkmark \\[+2ex]
-Prisbillig i forhold til kommercielle BI-sensorer & \checkmark \\[+2ex]
-SRM kan monitorere synkefrekvensen  & \checkmark \\[+2ex]
-SRM detektere synkefrekvensen & \checkmark \\[+2ex]
-Integreret kommerciel EMG-sensor & \checkmark \\[+2ex]
-Målingerne er simultane & \checkmark \\[+2ex]
-Brugergrænseflade til SRM  & \checkmark \\[+2ex]
-Ingen målinger fra personer med Dysfagi & $\div$ \\
\hline
\end{tabular}
\end{large}
\end{center}




