\chapter{Resultater}

I denne del af rapporten fremlægges projektets resultater fra accepttesten.  I accepttesen er der testet de funktionelle og ikke-funktionelle krav, der er defineret i \nameref{bilag1}. De observerede resultater i denne test præsenteres kort ved anvendelse af tabeller. 

\section{Funktionelle krav}

De funktionelle krav som dette projekt har prioriteret højst er de krav, der er defineret i \textit{Must og Should}. Til opfyldelsen af disse krav, er der benyttet use cases, der sammen definerer, hvordan disse krav opnås. De detaljeret testudførsel og resultater kan læses i \nameref{bilag6}. I tabel \ref{AT_UC1} præsenteres use casens resultat udfald. 
  
\begin{longtabu} to \linewidth{@{} c X[l] X[l] X[j] c@{}}
    ~ &	\textbf{Use case nr.} &    \textbf{Krav type} &		\textbf{Testresultat} \\[-1ex]
    \midrule
    ~ & 1 & Funktionelle krav & Godkendt &
    \\ \midrule
   &   2 &   Funktionelle krav & Godkendt   &	
   
\\ \midrule
   &   3 &   Funktionelle krav & Godkendt   &   
   
 \\ \bottomrule
 
\caption{Resultaterne for de funktionelle krav, der er defineret i kravspecifikationen}\\
\label{AT_UC1}
\end{longtabu}


De tre use cases tilsammen opfylder kravene i \textit{Must og Should} i kravspecifikationen med undtagelse krav nummer 9. Se kapitel 7, hvorfor dette krav ikke er opfyldt. 

\pagebreak
\section{Ikke-funktionelle krav}

Tilforskel fra de funktionelle krav er de ikke-funktionelle krav ikke organiseret i use cases. De ikke-funktionelle krav består af punkter, der definerer produktets kvalitetsaspekter. Disse skal være testbart. Herunder opsummeres resultaterne af accepttesen for de ikke-funktionelle krav. 

\begin{longtabu} to \linewidth{@{} c X[l] X[l] X[j] c@{}}
    ~ &	\textbf{Krav nr.} &    \textbf{Krav type} &		\textbf{Testresultat} \\[-1ex]
    \midrule
    ~ & 1 & ikke-funktionelle krav & ej testbart &
    \\ \midrule
   &   2 &   ikke-funktionelle krav & ej testbart   &	
   
\\ \midrule
   &   3 &   ikke-funktionelle krav & ej testbart   &   
   
   \\ \midrule
   &   4 &   ikke-funktionelle krav & Godkendt   &  
   
   
    \\ \midrule
   &   5 &   ikke-funktionelle krav & Godkendt   & 
   
    \\ \midrule
   &   6 &   ikke-funktionelle krav &  ej testbart  & 
   
   
    \\ \midrule
   &   7 &   ikke-funktionelle krav &  ej testbart  & 
     \\ \midrule
   &   8 &   ikke-funktionelle krav & Godkendt  & 
     \\ \midrule
   &   9 &   ikke-funktionelle krav &  ej testbart  &  
   \\ \midrule
    &   10 &   ikke-funktionelle krav &  Godkendt  &  
    
    \\ \midrule
    &   11 &   ikke-funktionelle krav &  Godkendt  &  
   
   
 \\ \bottomrule
 
\caption{Resultaterne for de ikke-funktionelle krav, der er defineret i kravspecifikationen}\\
\label{AT_UC2}
\end{longtabu}

Som det ses i tabel \ref{AT_UC2} er der en del krav, der er mærkeret som ej testbart under accepttesten. Gruppen har været fuld bevidst om at disse krav ikke er testbart, men har alligevel medtaget som et ønsket kvalitets parametre. Dette valg er truffet for at gøre brugeren/kunden opmærksom på, at produktet ikke er testet under de nævnte forhold og dermed skal brugeren være opmærksom på at anvende produktet under disse forhold, der ikke testet.  