\chapter{Konklusion}


Formålet med bachelorprojektet var at udvikle et device bestående af en prisbillig bioimpedans måler og EMG måler. Devicet skulle tilsammen kunne monitorere og detektere synkefrekvensen hos personer der er ramt af dysfagi. Dette device har resulteret i et proof-of-concept system kaldet SRM. Bachelorprojektet startede med en analyse af baggrunden for Dysfagi, og samtidig opbygning af bioimpedans måler. Dette resulterede i en række krav som blev prioriteret vha. MoSCoW-metoden. SRM har gennem acceptesten vist sig at opfylde alle Must krav samt de fleste Should krav i kravspecifikationen. Dog var det ikke muligt at validere SRM op imod en kommerciel BI måler, selvom gruppen var besiddelse af denne.

Resultaterne fra test af SRM kan det konkluderes at det var muligt at monitorere både BI og EMG signaler simultant, samtidig mens synk detekteres vha. algoritme. Det var ikke muligt at foretage test på personer med dysfagi, da der ikke ønskes at påføre yderligere traume på et i forvejen ramt vævssegment. De resultater SRM har kunne producere har været til tider fine og andre gange for ringe. Det skyldes bl.a. at SRM er implementeret på et fumlebræt, gør at det kan være svært at reproducere resultater, da det er følsomt over for berøring og støj fra nærliggende kilder såsom computer.


Bachelorprojektet har omhandlet en del hardware udvikling. Dette resulterede i en projektstart, hvor der var mange ting som skulle undersøges i hvordan BI måles og bygges elektroteknisk. Det at have mulighed for at bygge videre på de tidligere semesters læring, hjalp dog en hel del. Bachelorprojektet har givet et større kendskab om hardware og udviklingen af software til at styre en måleenhed. 

Der er dog stadig en del yderligere udviklingsarbejde for at SRM bliver mere stabil og nøjagtig. SRM vil kunne senere blive anvendt til klinisk brug på patienter som et objektiv beslutningsunderstøttene værktøj for klinikeren.












