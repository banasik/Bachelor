\chapter{Konklusion}


Dette bachelorprojekt har resulteret i et proof-of-concept system kaldet SRM. SRM består af en prisbillig bioimpedans måler som kan bruges som et detektionsværktøj, der kan detektere synk. SRM måler efter synk både fra BI perspektiv men også fra EMG perspektiv, der tilsammen giver større detekterbarhed af synk. Alle målinger foretages og vises i selv samme GUI. Softwaren kan, analysere både BI og EMG signalerne, vise begge signaler simultant, gemme signalerne og hente tidligere  målinger frem. SRM har gennem acceptesten vist sig at opfylde alle Must krav samt de fleste Should krav i kravspecifikationen. Dog var det ikke muligt at validere SRM op imod en kommerciel BI måler, selvom gruppen var besiddelse af denne. De resultater SRM har kunne producere har været til tider fine og andre gange for ringe. Det at SRM er bygget på et fumlebræt, gør at det kan være svært at reproducerer resultater, da det er følsomt over for berøring og støj fra nærliggende kilder såsom computer. 

Bachelor projektet har omhandlet en del hardware udvikling. Dette resulteret i en start af projektet, hvor der var mange ting som skulle undersøges i hvordan BI måles og bygges. Det at have mulighed for at bygge videre på de tidligere semesters læring, hjalp dog en del.




Der er dog stadig en del yderligere udviklingsarbejde for at SRM for at gøre det mere stabilt og  mere nøjagtigt nok til at senere kunne blive anvendt til klinisk brug på patienter.






