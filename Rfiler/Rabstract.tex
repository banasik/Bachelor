\chapter{Abstract}


\textbf{Background}

Dysphagia is a secondary disease in apoplexy patients. 30\% of this patient group has dysphagia. An incidence of dysphagia is also observed in Parkinson and Alzheimer's patients. People who are suffering from dysphagia are at greater risk of being affected with lung disease, which may result in death. Current screening of dysphagia patients contains subjective elements, which can lead to poorer treatment.

The purpose of this bachelor project is to develop a proof-of-concept product that can eventually become a support tool that can provide objective assessments of the swallowing   process.

\textbf{Materials and methods}

During this project, a proof-of-concept product has been developed that can detect and evaluate the rate of swallowing, over pharynx, in healthy subjects. The product consists of a self-developed affordable bioimpedance sensor that is combined with a commercial EMG sensor. Together the two sensors measure muscle activity and bioimpedance
in the measuring area. During the development of this project, the development tools ASE model and V model have been used to lead the project towards a finished prototype.

\textbf{Results}

The achieved system is tested on healthy objects that have a normal swallowing function. The result shows that the system is able to detect both the muscular electrical activity and the voltage that the injected current has resulted, in the tissue. The measured results are more or less consistent with what can be found in the literature. The system has on several occasions been unstable since it is difficult to accurately recreate the measurements.

\textbf{Conclusion}

Through this project, we have succeeded in developing a proof-of-concept product that can detect and monitor the rate of swallowing in healthy objects. Even though the product is functional, it is necessary to further develop it to reduce or eliminate the unstable outcomes observed during the developmental process. To improve this product, there have been proposed a number of areas of action that are obvious to include in future developmental work.
