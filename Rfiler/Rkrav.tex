\chapter{Krav} \label{Af:Krav}

På baggrund af et møde med Jim Jensen fra Hammel Neurocenter, som er projektets udbyder, er der udarbejdet en kravspecifikation til et system kaldet Synkerefleksmonitor (SRM). På mødet blev der diskuteret udfordringer med nuværende behandlinger og udredninger af dysfagipatienter. Formålet med mødet var ikke at etablere kunde/leverandør-relation, hvor krav til et kommende produkt skal forhandles på plads. Mødet havde i stedet en uformel karakter, hvor Hammel Neurocenter frivilligt er gået med til at mødes med gruppens medlemmer for at bidrage med deres ekspertise indenfor behandling og udredning af dysfagipatienter. Kravene til produktet som skal realiseres under dette projekt, er suverænt udspecificeret af gruppens medlemmer uden indblanding af projektets udbyder. Fra udbydernes side var der kun et ønske om at bidrage med udvikling af nye metoder til udredning af dysfagipatienter, hvilken dette projekt også har intentioner om. \\

I det følgende beskrives kort det overordnede system, efterfulgt af funktionelle og ikke funktionelle krav.  

\section{Systembeskrivelse}



Figur \ref{fig:sysbeskrivelse} illustrer systembeskrivelsen for SRM'en. Systemet igangsættes af et sundhedspersonale, ved at tilkoble elektroder fra SRM'en til et måleobjekt. Ved hjælp af en funktionsgenerator i systemet sendes en konstant strøm til måleobjektet via. elektroderne. På måleobjektets måleregion måles spændingsfaldet, som opstår, når forholdet mellem spænding og strøm ændrer sig. De målte spændinger omdannes til digitale værdier vha.  en A/D-konverter. Til slut vises disse værdier på en PC-skærm i form af en graf. Sundhedspersonalet har hermed mulighed for at evaluere måleobjektets synkefrekvens. I SRM'en eksisterer der delsystemer. Detaljer omkring, hvordan disse er byggede og designede, henvises der til \nameref{bilag4} og \nameref{bilag5}. 

 


 

\begin{figure}[H]
\centering
{\includegraphics[width=10cm]
{Figure/AktoerKontextDiagram}}
\caption{Aktør-kontekst diagrammet illustrer det overordnet system. Her ses to aktører, der direkte interagerer med SRM'en. Den tredje aktør er en offstage-aktør, der kommunikerer med systemet igennem elektrodeaktør.      }
\label{fig:sysbeskrivelse}
\end{figure}  

 

\section{Aktørbeskrivelse}
Til aktør-kontekst diagrammet følger der en en aktør beskrivelse, der beskriver kort hver aktørs funktion, se tabel \ref{tab:aktoerbeskrivelse}.    
\begin{table}[H]
\begin{tabularx}{\textwidth}{l l X}
     Aktørnavn	&	Type		&	Beskrivelse \\ \midrule
     Sundhedspersonale   	&  	Primær  	& 	Sundhedspersonalet tilkobler SRM'en til måleobjektet vha. elektroder, samt starter og evaluerer målingerne.     \\ 			  \addlinespace[2mm]
     Elektroder	&	Sekundær	& Elektroder anvendes til at måle bioimpedans og emg signaler fra måleobjektet  	 \\   \addlinespace[2mm]

Måleobjekt	&	Offstage-aktør	& Måleobjektet   leverer biosignaler, som elektroderne kan måle.   	 \\   \addlinespace[2mm]
     
   
     \bottomrule                                                                                                                   
    \end{tabularx}
    \caption {Aktørbeskrivelse for det samlede system}
    \label{tab:aktoerbeskrivelse}
	
\end{table}

\pagebreak
\section{Funktionelle krav}

Tabel \ref{tab:moscow} beskriver de funktionelle krav, der stilles til applikationen SRM'en. Nogle krav er vigtigere end andre, og de prioriteres vha. MoSCoW-metoden. Kravene i \textit{Must og Should} kategorien prioriteres højest.  I dette projekt bestræbes først og fremmest at opfylde kravene i \textit{Must og Should}. 

\begin{table}[H]

\begin{tabularx}{\textwidth}{X|X}
\rowcolor{Gray}
\toprule
\textbf{Must have} & \textbf{Should have} \\
\hline \\
\textbf{1. }Systemet skal have en bioimpedans sensor (BI), der kan måle bioimpedans signaler & \textbf{7. }Matlab GUI, der kan præsentere BI og EMG signaler \\[4ex]
\textbf{2. }Systemet skal have EMG sensor, der kan måle EMG signaler & \textbf{8. }Både BI og EMG målinger skal køre simultant\\[4ex]
\textbf{3. }Systemet skal kunne vise BI og EMG signaler over tid på en graf (offline) i Matlab  & \textbf{9. }Validere bioimpedans sensoren op imod kommerciel BI-sensor \\[4ex]
\textbf{4. }Systemet skal kunne beregne BI på baggrund af målte spændinger & \\[4ex]
\textbf{5. }Systemet skal kunne gemme BI og EMG målingerne & \\[4ex]
\textbf{6. }Systemet skal kunne hente tidligere BI og EMG målinger & \\[4ex]


\midrule
    \rowcolor{Gray}
    \textbf{Could have} & \textbf{Would have}\\
    \midrule \\
    \textbf{10. }Real-time visning af EMG- og BI signalerne & \textbf{14. }Mobilt synkerefleksmonitor med touch skærm\\[4ex]
\textbf{11. }Machine Learning for at diskriminere mellem synkerefleks og støj (tale og hoste) & \textbf{15. }EMG og BI signalerne overføres til EPJ  \\[4ex]
\textbf{12. }Systemt skal udføres på et færdigt print& \textbf{16. }Tage højde for anatomiske forskelle mellem kønnene\\[4ex]
& \\
\textbf{13. }Systemet gemmer BI og EMG målingerne i en database& \\[4ex]
& \\

\end{tabularx}

\caption{MoSCoW opdeling af kravene til  synkerefleksmonitoren.}
  \label{tab:moscow}
\end{table}


%\begin{table}[H]
%
%\begin{tabularx}{\textwidth}{X|X}
%\rowcolor{Gray}
%\toprule
%\textbf{Must have} & \textbf{Should have} \\
%\hline \\
%\textbf{1. }Systemet skal have en bioimpedans sensor (BI), der kan måle bioimpedans signaler & \textbf{5. }Matlab GUI, der kan præsentere BI og EMG signaler \\[4ex]
%\textbf{2. }Systemet skal have EMG sensor, der kan måle EMG signaler & \textbf{6. }Både BI og EMG målinger skal køre simultant\\[4ex]
%\textbf{3. }Systemet skal kunne vise BI og EMG signaler over tid på en graf (offline) i Matlab  & \\[4ex]
%\textbf{4. }Systemet skal kunne beregne BI på baggrund af målte spændinger & \\[4ex]
%
%
%\midrule
%    \rowcolor{Gray}
%    \textbf{Could have} & \textbf{Would have}\\
%    \midrule \\
%    \textbf{7. }Validere bioimpedans sensoren op imod kommerciel BI måler & \textbf{10. }Mobilt synkerefleksmonitor med touch skærm\\[4ex]
%\textbf{8. }Real-time visning af EMG- og BI signalerne & \textbf{11. }EMG og BI signalerne overføres til EPJ  \\[4ex]
%\textbf{9. }Machine Learning for at diskriminere mellem synkerefleks og støj (tale og hoste)& \textbf{12. }Tage højde for anatomiske forskelle mellem kønnene\\[4ex]
%& \\
%
%\end{tabularx}
%
%\caption{MoSCoW opdeling af funktionelle krav til  synkerefleksmonitorens software og hardware}
%  \label{tab:moscow}
%\end{table}

\pagebreak
For at realisere \textit{Must og Should } kravene, er der udviklet use cases, der bidrager til opfyldelsen af disse krav. Brugergrænsefladen til applikationen er forsøgt forsimplet, således at brugeren kun behøver at forholde sig til behandlede data og ikke rådata. På Figur \ref{UseCaseV1} ses de to aktører, der interagerer med tre use cases.  Herunder beskrives funktionaliteten af disse use cases. 


\begin{figure}[H]
\centering
{\includegraphics[width=10cm]
{Figure/usecasediagram}}
\caption{Use case diagram for synkerefleksmonitoren. Systemet består af tre usecases, der tilsammen bruges til at starte, måle, oploade tidligere målinger og gemme BI og EMG målinger}
\label{UseCaseV1}
\end{figure}

\textbf{Use Case 1 - Start Measurements:}
Denne use case bruges bl.a. til at igangsætte en måling. Når sundhedspersonalet interagerer med use casen, køres der en række funktioner, der tilsammen, starter, måler, behandler og visualisere to målinger. I  \nameref{bilag6} kan der bl.a. læses om de underliggende funktioner, som eksekveres, når denne use case aktiveres af brugeren.\\

\textbf{Use Case 2 - Save Measurements:}
Denne use case muliggør at brugeren kan gemme to målinger lokalt på computeren i en CSV-fil. Her gemmes de to målinger, som brugeren har taget simultant i use case 1. \\
 
\textbf{Use Case 3  - Load Measurements:} 
Denne use case tillader indlæsning af tidligere målinger, som er gemt lokalt på pc'en.  
 
 
 
 
 
 \pagebreak
Nedenunder er der en fully dressed for \textit{"Start Measurements"} use casen. Her er der beskrevet de trin som brugeren skal igennem for at tage en måling. Fully dressed for de resterende use cases henvises der til \nameref{bilag1}. 

\textbf{Use Cases - fully dressed  } 


\begin{longtabu} to \linewidth{@{}l r X[l]@{}} %UC1%
	{\large \textbf{Use Case - Start Measurements }} && \\
	\toprule
	Scenarie 				&&	Hovedscenarie\\
	Navn 					&& 	Start Measurements\\
	Mål 					&& 	At måle to signaler simultant \\
	Initiering 				&& 	Startes af Sundhedspersonalet\\
	Aktører 				&& 	Sundhedspersonale (primær), Elektroder (sekundær)\\
	
	Samtidige forekomster  	&& 	1 måling pr. kørsel \\
	Forudsætninger 			&&	BI-sensoren og EMG-sensoren er ledige og operationelle. Elektroderne påsat måleobjektet og GUI-vinduet er åbent \\ 
	Resultat 				&& 	To målinger foretages og vises til sundhedspersonalet\\ \midrule
	Hovedscenarie 			&    1. 	&	Sundhedspersonalet trykker på knappen "Start Measurements "\\				 	
							&    2. 	& 	En række funktioner køres automatisk og systemet sørger for at to målinger foretages simultant\\
	
	
	&&[\textit{Undtagelse 2.a:}] Systemet foretager ikke målinger\\ \midrule						
							
							
	Undtagelser 			&		2.a	& 	Applikationen genstartes og hovedscenarie 1 i use casen gentages \\ \bottomrule
	
	\caption{Fully dressed for use casen \textit{Start Measurements}}
	\label{UC1}
\end{longtabu}

\pagebreak
\section{Ikke-funktionelle krav}

Hvor de funktionelle krav kortlægger applikationens specifikke adfærd, fokuserer ikke-funktionelle krav på applikationens kvalitetsaspekter \cite{ISELektioner}.
Der benyttes en model kaldet $(F)URPS$ til at sætte ord på, hvad kunden, i dette projektets tilfælde gruppen, opfatter som kvalitet. Ved brug $(F)URPS$ skal man tage stilling til produktets funktionalitet, brugervenlighed, pålidelighed, ydeevne og supportering \cite{ISELektioner}. Man kan vælge at udvide  modellen til  $(F)URPS^{+}$. Her skal man yderligere forholde sig til flere kvalitetsbegreber. Lige som i de funktionelle krav er de ikke-funktionelle krav prioriteret ved brug af MoSCoW-metoden. Der er valgt at inddrage kun \textit{must-kategorien}. 

I dette bachelorprojekt anvendes kun $(F)URPS$ med undtagelse af  elementet funktionalitet. Det vurderes af gruppen at  dette element er afdækket i de funktionelle krav, og derfor er der ingen grund til at gentage det under de ikke-funktionelle krav. De resterende krav der stilles til produktet under ikke-funktionelle krav fokuserer på kvaliteter, der er vurderet nødvendige for produktets bruger. 

Nedenunder ses et udvalg af de krav, der  er stillet under $(F)URPS$ elementerne brugervenlighed, pålidelighed, ydeevne og supportering. 

\begin{itemize}
\item Sundhedspersonalet skal kunne aflæse et synk på graferne fra GUI'en på 2 meters afstand

\item Det skal maksimalt tage 10 minutter(0.17 timer) at gendanne Synkerefleksmonitor (MTTR - Mean Time To Restore)

\item GUI'ens objekter skal maksimum have en responstid på 30 sekunder. 

\item Sundhedspersonalet skal kunne udskifte elektroderne inden for 4 minutter.
\end{itemize}

Det skal oplyses at under de ikke-funktionelle krav er der medtaget nogle krav, som er svære at teste under dette projektets rammer. Herunder ses et udvalg af dem:

\begin{itemize}
\item  Synkerefeksmonitorens hardware skal kunne fungere under et temperaturforhold, der ligger mellem 10 til 40°C.
\item Synkerefleksmonitor skal have en oppetid uden nedbrud på minimum 1 dag (24 timer) (MTBF - Mean Time Between Failure)
\end{itemize}

Selvom disse krav som nævnt ikke kan testes, er det valgt at inkludere dem, da det er vigtigt at oplyse kunden, at produktet ikke er testet under de nævnte forhold. I \nameref{bilag1} kan man læse de detaljerede ikke-funktionelle krav.

 


 

