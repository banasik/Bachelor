\chapter{Fremtidigt arbejde}

Dette kapitel beskrives og prioriteres en række fremtidige indsatsområder, som er oplagte at arbejde videre med i en videreudvikling af SRM'en. Disse indsatsområder består af forbedringer af produktet og færdiggørelse af de krav, som står specificeret i projektets kravspecifikation, men ikke er nået implementeret.  


\section{Færdiggørelse af de funktionelle krav}
\textit{Could og Would} kravene i dette projekt er nedprioriteret i forhold til \textit{Must og Should} kravene, da det er vurderet at disse krav ikke er kritiske for at realisere en funktionsdygtig SRM. Implementering af disse krav vil for eksempel øge muligheden for at diskriminere, vha. machine learning, imellem bevægelse, der er forårsaget af et synk eller andre bevægelser som hovedbevægelser eller hoste. Implementering af et sådan krav vil også øge målingernes validitet og reliabilitet. En videreudvikling af SRM'en burde derfor indeholde opfyldelsen af \textit{Could og Would} kravene.   

\section{Færdiggørelse af de ikke-funktionelle krav}
Der er under de ikke-funktionelle krav medtaget en række krav som under accepttesten er bedømt som ikke testbart. I en videreudvikling af SRM'en bør derfor indeholde realisering og forbedring af disse ikke testbare krav. Hvis disse krav bliver gjort testbare vil det betyde at SRM'ens kvalitet bliver forbedret.    

\section{Risikoanalyse}
Risikoanalyse spiller en essentiel rolle ved udviklingen af medicinsk udstyr, da den belyser potentielle farer, der er forbundet med produktet. Hvis SRM'en skal bruges til kliniske undersøgelser er det nødvendigt at produktet går igennem en risikoanalyse efterfulgt af en risikohåndteringsplan. Man kan med fordel følge de harmoniserede standarder, der gælder indenfor området. Ved at følge en standard, kan man være sikker på at SRM'en opfylder de love, der gælder for medicinsk udstyr. Der er hverken, i dette projekt, foretaget en risikoanalyse eller risikohåndtering. Dette diskvalificerer SRM'en til at blive anvendt i kliniske undersøgelser. Det skal også nævnes, at det aldrig har været planen at udføre sådan nogle analyser, da SRM'en kun er et proof-of-concept, men dette er uundgåeligt, hvis produktet skal bruges som et medicinsk udstyr. Derfor bør en videreudvikling af dette produkt indeholde både en risikoanalyse og en risikohåndteringsplan.    
 
 
\section{Algoritme forbedringer}

Et indsatsområde, som er oplagt at arbejde med i en videreudvikling af SRM'en er at forbedre behandlingen af de målte signaler. Den udviklede algoritme til databehandling under dette projekt fokuserer på kun basale funktionaliteter, der skal til for at behandle et signal. En forbedring af algoritmen kan bl.a medføre reducering hardware komponenterne på fumlebrættet, ved at fjerne EMG-sensoren og nøjes kun med BI-kredsløbet. Dette betyder ikke at man eksludere EMG-signalet. Det man i stedet gøre er at benytte BI-kredsløbet til at måle både BI og EMG. Dette kan lade sig gøre, ved at man under databehandlingen af det målte signal, vha. BI-kredsløbet, udtrækker både BI og EMG  fra det samme dataset. Denne teknik er forsøgt eksperimenteret kort under dette projekt, men droppet efterfølgende til fordel for den nuværende løsning. Yderligere beskrivelse om den nævnte teknik henvises til artiklen  \cite{Nahrstaedt2012}.      

\section{Flere tests}
Under udviklingen af  SRM'en har der i flere omgange været uforklarlige udfald, hvor SRM'en ikke kunne detektere et synk. Om disse udfald skyldes systemet eller interfacet mellem systemet og måleobjektet, burde flere tests afdække. Der er også brug for at teste SRM'en på flere raske objekter, hvor man variere måleobjekternes køn, BMI og race. Der er kun foretaget tests på raske objekter under udvikling af SRM'en, hvilken ikke er nok til at vurdere, hvordan SRM'en virker på større populationer. Det er derfor nødvendigt at teste systemet mere for at identificere de uforklarlige udfald, der er omtalt.
 
 
\section{Validering}

SRM'en skal valideres op imod kommerciel BI-sensor for at verificere, om SRM'en er troværdig. Gruppen har forsøgt, som beskrevet i kapitel 3, at validere SRM'en op imod en kommerciel BI-sensor uden held. Dette bør ikke opgives, da sådan en validering kan give en indikation om SRM'en fungerer efter hensigten. Derfor vil det være oplagt at sammenligne de to apparater ved videreudvikling af SRM'en. 

\section{Wearable device}
En oplagt mulighed til en fremtidig videreudvikling er også at gøre SRM'en wearable. Første trin i den retning vil være at opbygge BI-kredsløbet med dets tilhørende komponenter på et print. Dette betyder bl.a. at Analog Discovery's opgaver i systemet flyttes til printet ved at man bygger delesystemerne som funktionsgeneratoren og ADC'en på printet. Fordelen ved et sådan en wearable device er at man ikke behøver at begrænse brugeren af systemet til en målestation.   


\section{Anvendelsesmuligheder}
En fremtidig anvendelsesmulighed i SRM'en kan være at man fremover kan få et apparat, der kan erstatte eller supplere de nuværende apparater, der bruges til at undersøge syne-spisevanskeligheder på dysfagipatienter. En anden anvendelsesmulighed er at brugeren kan tage devicet med hjem og måle sig selv uden professionel supervision. Devicet vil også kunne bruges på patienter med trachealtube. SRM'en kan være med til at evaluere, hvornår trachealtuberen kan fjernes. Disse anvendelsesmuligheder kan kun blive realistiske, hvis produktet forbedres med bl.a de nævnte indsatsområder i dette kapitel.     