\chapter{Resumé Abstract}

\textbf{Baggrund}\\
Dysfagi er en gennemgående følgesygdom for apopleksipatienter. 30\% af denne patientgruppe har dysfagi. Forekomsten af dyfagi observeres også hos Parkinson og Alzheimer patientgrupperne. Personer der er ramt af dysfagi har en større risiko for at blive ramt af lungebetegnelse, og kan resultere med døden til følge. Screening af dysfagipatienter er præget af subjektive vurderinger, som medfører dårligere behandlingsforløb. Formålet med dette bachelorprojekt er at udvikle et proof-of-concept produkt, der på sigt kan blive støtteværktøj, der kan give objektive vurderinger af synkefunktionen.

\textbf{Materialer og metoder}\\
Der er under dette projekt udviklet et proof-of-concept produkt, der kan detektere og evaluere synkefrekvensen, over pharynx, på raske personer. Produktet består af en selvudviklet prisbillig bioimpedans sensor, der er kombineret med en kommerciel EMG sensor. Begge sensorer implementeres på et fumlebræt. De to sensorer tilsammen måler  muskelaktiviteter og bioimpedans på måleregionen.   Under udviklingen af dette projekt er der brugt udviklingsværktøjerne ASE-modellen og V-modellen til at styre projektet hen imod en færdig prototype.  


\textbf{Resultater}\\
Det realiseret system er testet på raske objekter, der har en normal synkefunktion. Resultatet viser at systemet er i stand til at detektere musklernes elektriske aktivitet og spændingen som den injicerede strøm har resulteret i vævet simultant. De målte resultater stemmer nogenlunde overnes med, hvad der er fundet i litteraturen. Systemet har under flere lejligheder vist sig at være ustabil, da det har svært ved at genskabe målingerne nøjagtigt. 

\textbf{Konklusion}\\


