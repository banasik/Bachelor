\chapter{Resumé Abstract}

\textbf{Baggrund}\\
Dysfagi er en gennemgående følgesygdom for hjerneskadet. 30 procent af alle apopleksipatienter har dysfagi, men  patientgrupperne Parkinson og Alzheimer er også ramt. Med dysfagi er der stor risiko for lungebetegnelse og kan resultere med døden til følge. Screening for dysfagi bager præg af subjektive vurdering fra klinkeren. Formålet med bachelorprojektet er give klnikeren et støtte værktøj til at bestemme synkefrekvensen hos patienten.

\textbf{Materialer og metoder}\\
Denne projektrapport omhandler udviklingen af et proof-of-concept system, som kan detektere og evaluere synkefrekvensen på raske personer over pharynx vha. en Bioimpedans-måler og en EMG-måler. Denne information kan bruges til at sige noget om den kliniske tilstand. Der er kun udviklet på Bioimpedans-måleren som er en prisbillig sensor, hvor EMG-måleren er et kommercieltprodukt der bare implementeres. Under udviklingen af Bioimpedans-måleren er der brugt udviklingsværktøjerne ASE-modellen og V-modellen til at styre hen imod en færdig prototype. Undervejs i udviklingsprocessen, på baggrund af kravspecifikation, er der gjort overvejelser om arkitektur, design og tests. 


\textbf{Resultater}\\



\textbf{Perspektivering}\\

Emne





 Dette har resulteret i en godkendt Acceptest



afgrænsning


synsvinklen



En kort oversigt over den faglige gennemgang og konklusionen. Det bør indledes med en meget kort redegørelse for emnet, afgrænsningen og synsvinklen. Formålet er kun at for-tælle læseren, hvad denne projektrapport kan bruges til.