\chapter{Afgrænsning}
Bachelorprojektet har først og fremmest fokus på at opfylde de funktionelle krav i \textit{Must og Should} kategorien for at realisere et proof-of-concept produkt, der er funktionsdygtigt. De resterende krav i \textit{Could have og Would have} prioriteres lavest, og der bruges tid på dem, hvis de to andre kategorier er opfyldt tilfredsstillende. Der udvikles, hverken use cases eller test cases for \textit{Could have og Would have} før de to andre er på plads og der i øvrigt er nok tid at arbejde med dem. For de ikke-funktionelle kraves vedkommende prioriteres kun de testbare krav højest, hvorimod de ikke testbare krav prioriteres lavest. \\


Undervejs i projektet er det konstateret at krav nummer 9 i de funktionelle krav ikke er muligt at opfylde. For at validere den udviklede BI-sensor med en kommerciel BI-sensor, har gruppen taget kontakt til Steven Brandtlov, fra Indkøb og Medicoteknik Skejby Sygehus, gennem  Peter Johansen, lektor på Institut for Ingeniørvidenskab på Aarhus Universitet. Etablering af kontakten har taget længere tid pga. manglende respons fra Steven Brandtlov side, se \nameref{bilag11}. Da gruppen endelig kom i besiddelse af BI-sensoren,  fungerede den ikke efter dens brugeranvisning. Efter flere mislykkede mail korrespondance med Brandtlov, blev gruppen enig om at henstille dette krav.      