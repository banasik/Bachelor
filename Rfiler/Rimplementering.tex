\chapter{Implementering}

\section{Hardware}
I dette kapitel beskrives kort, hvordan SRM'ens hardware og software komponenter er implementeret. Implementeringen afspejler designet, der er udarbejder under designfasen, se \nameref{bilag5}.  Til implementering af hardware komponenterne, er der valgt at illustrere en testopstilling ved hjælp af et diagram, inden kredsløbet bliver bygget på et fumlebræt. Denne testopstilling kan ses på figur \ref{fig:integrationstestDiagram}

\begin{figure}[H] 
\centering
{\includegraphics[width=\linewidth]
{Figure/integrationstestDiagram}}
\caption{Et diagram der illustrerer, hvordan hele systemets hardware komponenter skal implementeres på et fumlebræt. }
\label{fig:integrationstestDiagram}
\end{figure}

\pagebreak
Den overstående testopstilling blev efterfølgende bygget på fumlebrættet. På figur \ref{fig:integrationstestBilleder1} ses et fumlebræt, et multimeter,  BI-kredsløbet, EMG-sensoren, forsyningsspændingen til kredsløbet, elektrode ledninger til hhv. BI-kredsløbet og EMG-sensoren. Fumlebrættet er tilkoblet til AD'en, som bruges til både at forsyne BI kredsløbet med et AC signal og til dataopsamling. 


\begin{figure}[H]
\centering
{\includegraphics[width=12cm]
{Figure/SamledeSystem}}
\caption{Et foto af alle komponenter implementeret på et fumlebræt. }
\label{fig:integrationstestBilleder1}
\end{figure} 


Figur \ref{aaspectrumimplementering} viser delsystemer, som er bygget på fumlebrættet. Her er der forsøgt at gøre ledninger så korte som muligt. 

\begin{figure}[H] 
\centering
{\includegraphics[width=12cm]
{Figure/aaspectrumimplementering}}
\caption{Et nært billede af, hvordan SRM'en er bygget på et fumlebræt.}
\label{aaspectrumimplementering}
\end{figure}

\pagebreak


Det implementeret system på fumlebrættet kobles nu til måleobjektet vha. elektroder. På figur \ref{fig:maaleobjekt} ses de elektroder, der er tilkoblet til måleobjektet. Der er i alt brugt syv silver chloride( Ag/AgCl) elektroder som anvendes til både at transmittere strøm til måleobjektet og måle spændingsfaldet over måleregionen. Elektrode placeringen som ses på figur \ref{fig:maaleobjekt} er bl.a. inspireret af artiklen \textit{"Neck Electrical Impedance for Measurement of Swallowing"}.     

Fire af de i alt syv elektroder er koblet til BI-kredsløbet, hvor de resterende tre elektroder er koblet til EMG-sensoren.   
\begin{figure}[H] 
\centering
{\includegraphics[width=10cm]
{Figure/maleobjektmedelektroder}}
\caption{Her ses et foto af, hvordan elektroderne er koblet til måleobjektet. De yderste to elektroder på hver sin side benyttes til at transportere strøm(I) til måleobjektet. De to inderste to elektroder anvendes til at måle spændingsfaldet(V), når måleobjektet synker. De tre små elektroder, der ses øverste, bruges til at måle EMG.}
\label{fig:maaleobjekt}
\end{figure}

\pagebreak
\section{Software}

På baggrund af designfasen for SRM'ens software kan implementeringen påbegyndes. Til implementering af systemets software-del benyttes  udviklingsværktøjet/scriptsproget Matlab, version R2017b, pga. følgende fordele:


\begin{itemize}
\item Matlab forærer databehandlingsfunktioner som ligger klar til anvendelse
\item Matlab er god til at indlæse store mængde data med få kodelinjer
\item I Matlab kan man udvikle en brugergrænseflade nemt og hurtigt 
\item Ved brug af Matlab med Analog Discovery, kan man betjene  funktionsgeneratoren, dataopsamlingsenheden og brugergrænsefladen  fra ét sted
\end{itemize}
   
Med disse fordele var det oplagt at vælge Matlab som både udviklingsmiljø og programmeringssprog.  
Kodeimplementeringen er organiseret i små funktioner, som kan kaldes på tværs af hinanden. Implementeringen af koden tager højde for, at koden kan udvides uden at det påvirker andre dele af systemet. Ved en måling trykkes knappen "Start\_Measurements" hvilket kalder en række funktioner. Én af disse funktioner er funktionen \textit{"Process\_Measurements"}, som indeholder kodestykker, der behandler bl.a. BI-signalet.

%\lstinputlisting[frame=single, firstline=96, lastline=105]{matlabkode/Synkerefleksmonitor.m}


\lstinputlisting[frame=single, firstline=1, lastline=38]{matlabkode/Process_Measurements.m}

%\begin{figure}[H]
%\centering
%{\includegraphics[width=8cm]
%{Figure/designUML}}
%\caption{Figuren viser indholdet af kodeeksekvering i funktionen \textit{"Process\_Measurements"} i den del hvor der udføres envelope af det rå BI-signal. Først bliver det rå BI signal dobbeltensrettet og dernæst filtreret vha. digital lavpasfilter med knækfrekvens = 500 Hz og dæmper med 40 dB over en dekade. Til sidst bliver signalet detrended, downsampled og synk detekteres.}
%\label{Fig:designUML}
%\end{figure} 

Yderligere beskrivelse af de resterende Callbacks og funktioner, henvises der til \nameref{bilag6}. 
