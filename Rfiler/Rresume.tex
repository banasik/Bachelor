\chapter{Resumé}

\textbf{Baggrund}

Dysfagi er en gennemgående følgesygdom for apopleksipatienter. 30\% af denne patientgruppe har dysfagi. Forekomsten af dysfagi observeres også hos Parkinson og Alzheimer patienter. Personer der er ramt af dysfagi har en større risiko for at blive ramt af lungebetegnelse, hvilken kan forårsage dødsfald. Nuværende undersøgelsesmetoder af dysfagipatienter er præget af subjektive vurderinger, som kan medføre dårligere behandlingsforløb. 

Formålet med dette bachelorprojekt er at udvikle et proof-of-concept produkt, der på sigt kan blive et støtteværktøj, der kan give objektive vurderinger af synkefunktionen hos dysfagipatienter. 

\textbf{Materialer og metoder }

Der er under dette bachelorprojekt udviklet et proof-of-concept produkt, der kan detektere og evaluere synkefrekvensen, over pharynx, på raske personer. Produktet består af en selvudviklet prisbillig bioimpedans sensor, der er kombineret med en kommerciel EMG sensor. Begge sensorer implementeres på et fumlebræt. De to sensorer tilsammen måler muskelaktiviteter og bioimpedans på måleregionen. Under udviklingen af dette projekt er der brugt udviklingsværktøjerne ASE-modellen og V-modellen til at styre projektet hen imod en færdig prototype. 

\textbf{Resultater}

Det realiseret system er testet på raske objekter, der har en normal synkefunktion. Resultatet viser at systemet er i stand til at detektere musklernes elektriske aktiviteter og spændingen, som den injicerede strøm resulterer i vævet, simultant. De målte resultater stemmer nogenlunde overnes med, hvad der er fundet i litteraturen. Systemet har under flere lejligheder vist sig at være ustabilt, da det har svært ved at genskabe målingerne nøjagtigt. 

\textbf{Konklusion}

Det er gennem dette projekt lykkes at udvikle et proof-of-concept produkt, der kan detektere og monitorere synkefrekvensen på raske objekter. Selvom produktet er funktionsdygtigt, er det nødvendigt at videreudvikle det for at reducere eller eliminere de ustabile udfald, som er observeret under udviklingsprocessen. Til forbedring af dette produkt er der foreslået en række indsatsområder, som er oplagt at have med i et fremtidigt udviklingsarbejde.



