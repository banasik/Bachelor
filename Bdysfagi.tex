\documentclass[main.tex]{subfiles}
\begin{document}

\chapter*{Dysfagi}
\section*{Indledning}
Når man drikker eller og spiser skal dette forbi svælget og videre ned i spiserøret. Denne synkningsprocess foregår ved at den tygget mad skubbes bagud mod svælget, ved hjælp af tungen som presser op og bagud. Nu udløses en refleks pga. sanseceller i svælget. Dette foregår uden for viljens kontrol og i forbindelse med den forlængede rygmarv. Hele synkeprocessen er et samspil mellem 20-30 muskler og skal enten trække sig sammen eller slappe af på den rigtige rækkefølge. \cite{Sand2008MennesketsFysiologi}

Hvad er dysfagi?
Dysfagi opstår når man ikke får lavet en korrekt synkning, så i stedet for at det ender i spiserøret går det i stedet går ned i luftvejene. Som kan resultere i bl.a. lungebetændelse.\cite{Kjaersgaard2013DifficultiesPerspective}



Hvor opstår dysfagi? anatomi 
Hvor opstår dysfagi? sygdommen?

\bibliography{Mendeley.bib}
\end{document}